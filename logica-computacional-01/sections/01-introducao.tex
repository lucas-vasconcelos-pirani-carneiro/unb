\section{Introdução}

    \subsection{O que é Lógica ?}
    
        \begin{itemize}[left=0.5cm, align=left, nosep]
            \item É a ciência que estuda a \underline{validade dos argumentos} 
            \item Estuda os \underline{métodos} e \underline{pricípios} para comprovar os argumentos
            \item \textbf{Argumento :} É uma \underline{sequência de fatos} que é usado para concluir algo.   
            \begin{itemize}[left=0.5cm, nosep, label=$\hookrightarrow$]
                \item Pode ser expresso em \textbf{linguagem natural} ou \textbf{linguagem formal}
            \end{itemize}
            \item \textbf{Argumento Correto :} É um argumento em que os fatos justificam, sem falhas, uma 
            determinada conclusão
        \end{itemize}
   
    \subsection{Linguagem}
        \begin{itemize}[left=0.5cm, align=left, nosep]
            \item \textbf{Linguagem Natural :} Uso cotidiano
            \begin{itemize}[left=0.5cm, nosep, label=$\hookrightarrow$]
                \item Prolixa
                \item Ambiguidade
            \end{itemize}
        
            \item \textbf{Linguagem Formal :} \underline{Sistema simbólico} preciso e operacional de modo a evitar a ambiguidade e a loquacidade das linguagens naturais
            \begin{itemize}[left=0.5cm, nosep, label=$\hookrightarrow$]
                \item Concisa
                \item Exata
                \item Exemplos: Matemática, Música, Relógio, Linguagens de Programação, \dots  
            \end{itemize}
        
        \end{itemize}        

        \subsubsection*{Dimensões da Linguagem}
           
            \begin{enumerate}[left=0.5cm, align=left, nosep]
                \item \textbf{Sintaxe :} \underline{Ordem} de como são escritas as palavras
                \item \textbf{Semântica :} \underline{Significado} que as palavras possuem
                
                Exemplo : $2 + 2 = 4$
                \begin{itemize}[left=0.5cm, nosep, label=$\hookrightarrow$]
                    \item Estrutura matemática : sintaxe 
                    \item Valor : semântica
                \end{itemize}

                \item \textbf{Pragmática :} Sem relação com a sintaxe e semântica, significado atribuido por 
                questões históricas, sociais e culturais.
                
                Exemplo : Que horas são ?
            
            \end{enumerate}

        \subsubsection*{Metalinguagem}
            
            \begin{itemize}[left=0.5cm, align=left, nosep]
                \item \textbf{Metalinguagem :} É uma linguagem que explica qualquer outra linguagem   
                \item \textbf{Linguagem Objeto :} É a linguagem da qual se fala
                 
                $\text{Ex}_1$ : "House" é o mesmo que "casa"   
                \begin{itemize}[left=0.5cm, nosep, label=$\hookrightarrow$]
                    \item Metalinguagem : português
                    \item Linguagem Objeto : inglês
                    \item "\ "\ : Separar a metalinguagem da linguagem
                \end{itemize}
        
                \vspace{0.5cm}
                $\text{Ex}_2$ : $\int_{a}^{b} f(x) \,dx$   
                \begin{itemize}[left=0.5cm, nosep, label=$\hookrightarrow$]
                    \item Metalinguagem : português
                    \item Linguagem Objeto : matemática
                \end{itemize}
                
                \item \textbf{Hierarquia infinita de metalinguagem :} Para ter consistência na análise da linguagem é 
                preciso que uma linguagem de \underline{nível inferior} para explicá-la.
                
                \begin{itemize}[left=0.5cm, nosep, label=$\hookrightarrow$]
                    \item Ou seja, dada uma linguagem-objeto precisamos de uma metalinguagem para 
                    explicá-la, que por sua vez precisa de uma metametalinguagem para explicá-la, 
                    e assim sucessivamente. 
                \end{itemize}

                \vspace{0.5cm}

                \begin{center}
                    \begin{tikzpicture}[
                        node distance=0.5cm,
                        box/.style={
                            draw, 
                            rounded corners=3pt, 
                            fill=gray!10, 
                            minimum width=4cm, 
                            align=center, 
                            font=\sffamily\bfseries
                        }
                    ]
                    \node[box] (obj) {Objeto};
                    \node[box] (meta) [below=of obj] {Metaliguagem};
                    \node[box] (metameta) [below=of meta] {Metametaliguagem};
                    \node (dots) [below=of metameta] {$\ldots$};

                    \foreach \i/\j in {obj/meta, meta/metameta, metameta/dots} {
                        \draw[->, thick, shorten >=4pt, shorten <=4pt] (\i) -- (\j);
                    }
                    \end{tikzpicture}    
                \end{center}

                \vspace{-0.3cm}

                Exemplo : $2 + 2 = 4$ \\
                \textbf{Linguagem-objeto :} “$2 + 2 = 4$” — uma sentença da linguagem da \emph{aritmética}. \\
                \textbf{Metaliguagem :} “A expressão ‘$2 + 2 = 4$’ é verdadeira.” — frase na linguagem natural (Português) 
                    descrevendo a sentença da linguagem-objeto. \\
                \textbf{Metametalinguagem :} “A frase ‘A expressão $2 + 2 = 4$ é verdadeira’ é uma afirmação correta 
                    sobre a linguagem da aritmética.” — análise sobre a metalinguagem.
            
                \item \textbf{Teoria :} É um conjunto de explicações para descrever um fenômeno
                \item \textbf{Metateoria :} É a teoria que investiga, analisa ou descreve a própria teoria
            
            \end{itemize}
       
    \subsection{Uso e Mensão}
        
        \begin{itemize}[left=0.5cm, align=left, nosep]
            \item Relacionados diretamente com os \underline{níveis da linguagem} em que os termos aparecem.    
            \item \textbf{Usa-se} um termo para afirmar certas coisas.
            \item \textbf{Menciona-se} um termo quando falamos à respeito dele próprio.
            
            \begin{itemize}[left=0.5cm, nosep, label=$\hookrightarrow$]
                \item $\text{Ex}_1$ : Gato é um animal bonitinho
                \item $\text{Ex}_2$ : "Gato" tem cinco letras 
                \item $\text{Ex}_3$ : "Lucas" é um nome bíblico
            \end{itemize}
           
            \item \textbf{Número :} É um certo tipo de objeto matemático
            \item \textbf{Numeral :} É o nome de um número
            \item \textbf{Substituendos :} São expressões que podem ser colocados no lugar de variáveis
            \item \textbf{Valores da variáveis :} É o domínio em que a variável está inserida.

        \end{itemize}
    
        \subsubsection*{Exemplos}
            $\text{Ex}_1$ : “Rosa” é dissílaba. \\
            $\text{Ex}_2$ : Napoleão foi imperador da França. \\
            $\text{Ex}_3$ : A palavra “water” tem o mesmo significado que a palavra portuguesa “água” \\
            $\text{Ex}_4$ : “ “Logik” ” não pode ser usada como sujeito de uma sentença do português. \\
            $\text{Ex}_5$ : “Pedro” não é o nome de Sócrates, mas é o nome de “Pedro”. \\
            $\text{Ex}_6$ : O numeral “8” designa a soma de 4 mais 4. \\
            $\text{Ex}_7$ : 2+2 é igual a 3+1, mas “3+1” é diferente de “4” \\
            $\text{Ex}_8$ : A sentença nenhum gato é preto é falsa. A sentença “nenhum gato é preto” é falsa. \\
            $\text{Ex}_9$ : “Todavia” e “contudo”, mas, não também têm o mesmo que significado que “mas”, contudo, não, não. 
