\documentclass[a4paper,12pt]{article}
\usepackage[utf8]{inputenc}
\usepackage{amsmath,amssymb}
\usepackage[brazil]{babel}
\usepackage{geometry}
\geometry{margin=2.5cm}
\usepackage{enumitem} 

\pagestyle{empty}

\begin{document}

\begin{center}
    \large Álgebra 1
\end{center}

\begin{center}
    \large Lista 07 \\
    \small (Congruência Módulo $m$)
\end{center}

  \begin{enumerate}[label=7.\arabic*.]
    \item (Gavetas de Dirichlet) Se $x_1,\ldots,x_m$ são inteiros, demostre que a soma de um certo subconjunto
    de $\{x_1,\ldots,x_m\}$ é um múltiplo de $m$.

    (Dica: Considere as distintas classes módulo $m$ dentre as determinadas por $0, x, x_1 + x_2, \ldots , x_1 + \ldots + x_m$)

    \item (Quadrados perfeitos II). Verifique que cada quadrado perfeito é congruente a 
    $0, 1$, ou $4 \pmod{8}$.

    \item (Teorema Chinês dos Restos I). Dados inteiros $a$ e $b$, verifique que uma condição 
    necessária e suficiente para que o par de congruências $x \equiv a \pmod{m}$, 
    $x \equiv b \pmod{n}$ possua uma solução é que $a \equiv b \pmod{d}$, onde $d = \text{m.d.c}(m,n)$. Se $d = 1$, 
    verifique que a solução é única módulo $mn$.

    \item (Polinômios e divisibilidade). Se $f(X)$ é um polinômio com coeficientes inteiros 
    e se $x \equiv y \pmod{m}$, verifique que $f(x) \equiv f(y) \pmod{m}$. Obtenha critérios de 
    divisibilidade por $9$ e por $11$.

    \item (Módulo composto). Se $m$ e $n$ são inteiros positivos relativamente primos, e $f(X)$ 
    é um polinômio com coeficientes inteiros, demonstre que a congruência 
    $f(x) \equiv 0 \pmod{mn}$ possui uma solução se, e somente se, $f(x) \equiv 0 \pmod{m}$ e 
    $f(x) \equiv 0 \pmod{n}$ ambas possuem soluções. \\
    (Dica: use o Teorema Chinês dos Restos e o Exercício 7.4.)

    \item (Levantamento de Hensel). Sejam $p$ um primo e $f(X)$ um polinômio com coeficientes 
    inteiros. Suponha que $f(a) \equiv 0 \pmod{p^k}$ para algum inteiro $k>0$. Prove que as 
    seguintes condições são equivalentes:
    \begin{enumerate}
        \item[(i)] $f(a + tp^k) \equiv 0 \pmod{p^{k+1}}$ para algum $t$;
        \item[(ii)] $Df(a)t \equiv -\frac{f(a)}{p^k} \pmod{p}$ para algum $t$;
        \item[(iii)] ou $Df(a) \not\equiv 0 \pmod{p}$ ou $f(a) \equiv 0 \pmod{p^{k+1}}$.
    \end{enumerate}

    \item (Relação de equivalência). Para $x$ e $y$ inteiros $>0$, escreva $x \sim y$ se $\frac{y}{x}$ é uma potência 
    de $2$; prove que isso é uma relação de equivalência, e que $x \sim y$ se, e somente se, 
    os divisores ímpares de $x$ são os mesmos de $y$.

  \end{enumerate}

\end{document}
