\documentclass[a4paper,12pt]{article}
\usepackage[utf8]{inputenc}
\usepackage{amsmath,amssymb}
\usepackage[brazil]{babel}
\usepackage{geometry}
\geometry{margin=2.5cm}
\usepackage{enumitem} 

\pagestyle{empty}

\begin{document}

\begin{center}
    \large Álgebra 1
\end{center}

\begin{center}
    \large Lista 11
\end{center}
  
\begin{enumerate}[label=11.\arabic*.]

    \item (\textbf{Corpos finitos}). Se $F$ é um corpo finito, mostre que o subgrupo do grupo aditivo de $F$ gerado por $1$ possui ordem prima $p$, e é um subcorpo de $F$ isomorfo ao corpo $\mathbb{F}_p$ das classes de congruência módulo $p$. 

    \item (\textbf{Geradores}). Faça uma tabela de conjuntos minimais que geram $(\mathbb{Z}/m\mathbb{Z})^\times$ para $1 \leq m \leq 24$. 

    \item (\textbf{Grupos cíclicos I}). Sejam $G$ um grupo comutativo, $x$ um elemento de $G$ de ordem $m$ e $y$ um elemento de $G$ de ordem $n$. Prove que se $\gcd(m,n)=1$, então 
    \[
    x^a y^b = 1 \iff x^a = 1 = y^b,
    \] 
    donde conclua que o grupo gerado por $x$ e $y$ é cíclico de ordem $mn$, gerado por $xy$. 

    \item (\textbf{Grupos não cíclicos}). Demonstre que, se $m > 2$, $n > 2$ e $\gcd(m,n)=1$, então o grupo multiplicativo das classes de congruência primas a $mn$ módulo $mn$ não é cíclico. 
    \\ \textit{Dica: use o Teorema Chinês dos Restos e o fato de que um grupo cíclico possui no máximo um subgrupo de ordem $2$.} 

    \item (\textbf{Grupos cíclicos II}). Encontre todos os valores de $n$ para os quais o grupo multiplicativo das classes de congruência ímpares módulo $2^n$ é cíclico. 
    \\ \textit{Dica: se $a$ é ímpar, então $a^{2^{n-2}} \equiv 1 \pmod{2^n}$ para cada $n \geq 3$.}

    \item (\textit{$n$-torção de grupos abelianos}). Verifique que, se $G$ é um grupo comutativo e $n$ é um inteiro $>0$, então o conjunto de todos os elementos de $G$ cujas ordens dividem $n$ é um subgrupo de $G$. 

    \item (\textit{Elementos de ordem 2}). Prove que, se $G$ é um grupo comutativo finito, então o produto de todos os elementos de $G$ é ou $1$ ou um elemento de ordem $2$. Prove que, se $p$ é um primo, então 
    \[
    (p-1)! \equiv -1 \pmod{p}.
    \]

    \item (\textit{Grupos abelianos I}). Seja $G$ um grupo aditivo comutativo gerado por $\{x_1, \ldots, x_k\}$ e sejam $c_1, \ldots, c_k$ inteiros tais que $\mathrm{mdc}(c_1, \ldots, c_k) = 1$. Mostre que existe um conjunto $\{y_1, \ldots, y_k\}$ que gera $G$ tal que 
    \[
    y_1 = c_1x_1 + \cdots + c_kx_k.
    \] 
    \textit{Dica: suponha que cada $c_i \geq 0$ e proceda por indução sobre $s = c_1 + \cdots + c_k$.}

    \item (\textit{Grupos abelianos II}). Dentre todos os conjuntos minimais $\{x_1, \ldots, x_k\}$ que geram um grupo aditivo comutativo $G$, tome um deles tal que $x_1$ possua a menor ordem possível. Use o Exercício 11.8 para ver que não existem inteiros positivos $m_1, \ldots, m_k$ tais que 
    \[
    m_1x_1 + \cdots + m_kx_k = 0 \quad \text{com } m_1x_1 \neq 0.
    \] 
    Conclua, por indução, que existe $\{g_1, \ldots, g_k\}$ que gera $G$ tal que se 
    \[
    g = n_1 g_1 + \cdots + n_k g_k
    \] 
    com cada $n_i$ inteiro, então $n_i g_i = 0$ para cada $i$.

\end{enumerate}

\end{document}
