\documentclass[a4paper,12pt]{article}
\usepackage[utf8]{inputenc}
\usepackage{amsmath,amssymb}
\usepackage[brazil]{babel}
\usepackage{geometry}
\geometry{margin=2.5cm}
\usepackage{enumitem} 

\pagestyle{empty}

\begin{document}

\begin{center}
    Álgebra 1 
\end{center}

\begin{center}
    \large Lista 04 \\
    \small (Inteiros Coprimos e Primos)
\end{center}

\begin{enumerate}[label=4.\arabic*.]
    \item (Um produto pode dividir). Se m.d.c.$(a,b) = 1$ e ambos $a$ e $b$ dividem $c$, demonstre que $ab$ divide $c$.
    
    Se $mdc(a,b) = 1$, então $ \exists \text{ } x,y \in \mathbb{Z}$ tal que $ax + by = 1 (1)$

    Além disso, como $a \mid c$ e $b \mid c$ temos:
    
    $c = a k_1, k_1 \in \mathbb{Z}$ e $c = b k_2, k_2 \in \mathbb{Z}$ 
    
    Fazendo $c \cdot (1)$ : $c = cax + cby$

    Substituindo em $(2)$ e $(3)$ : 
    $c = ab(k_2x) + ab(k_1y)$

    $c = ab(k_2x + k_1y)$, como $k_2x + k_1y \in \mathbb{Z}$
    
    Logo, $ab \mid c$

    \item (Teorema da raiz integral). Demonstre que: se a equação $x^n + a_{n-1} x^{n-1} + \dots + a_0$ onde $n > 0$ e $a_0$ são inteiros, possui uma raiz racional, então esta raiz é um número inteiro. (Dica: use o Exercício 3.7.)
    
    Seja $x = \frac{p}{q}$ a racional dessa equação na forma mais simples tal que $mdc(p,q) = 1$.

    Substituindo essa raiz temos : 
    \begin{align*}
            (\frac{p}{q})^n + a_{n-1}(\frac{p}{q})^{n-1} + \ldots + a_0 = 0 (\cdot q^n) \\
            p^n + a_{n-1}(p)^{n-1}q + \ldots + a_0q^n = 0 \\ 
            p^n = - a_{n-1}(p)^{n-1}q + \ldots - a_0q^n \\
            p^n = q \cdot (- a_{n-1}(p)^{n-1} + \ldots - a_0q^{n-1})
    \end{align*}
    Assim, $q \mid p^n$

    Conteúdo, $mdc(p,q) = 1$ , ou seja, são primos entre si

    Então, a única maneira de $q$ dividir $p$ é se $q$ for $1$ ou $-1$, porém $q > 0$.

    Portanto, $q = 1 \Rightarrow x = \frac{p}{q} = p \in \mathbb{Z}$

    \item (Mersenne e Fermat). Seja $n$ um inteiro $>1$. Mostre que: 
    
    (i) se $2^n - 1$ é um primo, então $n$ deve ser primo; 
    
    Contrapositiva: Se $n$ é composto, então $2^{n-1} - 1$ deve ser composto
    
    Suponha que $n$ é composto, ou seja, $n = rs$, com $r \cdot s > 1$.
    
    Sabendo que $(x-y \mid x^k - y^k)$ temos que : $2^r - 1 \mid (2^r)^s - 1$, ou seja, $2^n - 1$ não é primo,
    \underline{contradição}

    Logo, $n$ é primo.

    (ii) se $2^n + 1$ é um primo então $n$ deve ser uma potência de 2. (Dica: se $d$ for um inteiro $>1$...)
    
    Suponha que $p \neq 2$ é um fator primo de $n$.Então, $n = n'p$ para algum $n' \in \mathbb{N}^*$.
    
    Como $p$ é ímpar segue que : $2^{n'} + 1 \mid (2^{n'})^p + 1^p = 2^n + 1$
    
    Logo, $2^{n'} + 1$ têm um divisor diferente de $1$ e de $2^n + 1$, \underline{contradição} pois $2^n + 1$ é primo

    Portanto, o único fator primo de $n$ é $2$,e $n = 2^m,\text{ com } n \in \mathbb{N}^*$
        
    \item (Valoração p-ádica de $n!$). Sejam $n$ um inteiro $\geq 1$ e $p$ um primo. Se, para cada número real $x$, denotarmos por $[x]$ o maior inteiro $\leq x$, prove que o maior inteiro $N$ tal que $p^N$ divide $n!$ é dado por $N = \sum_{k=1}^{\infty} \left[ \frac{n}{p^k} \right]$.
    
    

    \item (O polinômio $x^2 + x + 41$ não produz só primos). Prove que dentre os números representados pelo polinômio $a_n x^n + \ldots + a_0$, onde $n > 0$ e $a_0$ são inteiros com $a_n > 0$, existe uma infinidade de números não primos.

    

\end{enumerate}

\end{document}
