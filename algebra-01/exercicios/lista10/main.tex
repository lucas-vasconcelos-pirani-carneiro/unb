\documentclass[a4paper,12pt]{article}
\usepackage[utf8]{inputenc}
\usepackage{amsmath,amssymb}
\usepackage[brazil]{babel}
\usepackage{geometry}
\geometry{margin=2.5cm}
\usepackage{enumitem} 

\pagestyle{empty}

\begin{document}

\begin{center}
    \large Álgebra 1
\end{center}

\begin{center}
    \large Lista 10 \\
    \small (Anéis)
\end{center}
  
\begin{enumerate}[label=10.\arabic*.]

    \item (\textbf{A função $\varphi$ de Euler}). Se $m$ e $n$ são inteiros $> 0$, e $\gcd(m,n) = 1$, prove que 
    \[
    \varphi(mn) = \varphi(m)\varphi(n).
    \] 
    Se $p$ é um primo, prove que 
    \[
    \varphi(p^n) = p^n - p^{n-1}.
    \] 
    \textit{Sugestão: use o Teorema Chinês dos Restos para a primeira parte.} 

    \item (\textbf{$\mathbb{Z}[\sqrt{2}]$ e $\mathbb{Q}(\sqrt{2})$}). Verifique que os números da forma 
    \[
    x + y\sqrt{2}, \quad x,y \in \mathbb{Z},
    \] 
    formam um anel. Se $x,y \in \mathbb{Q}$, verifique que eles formam um corpo. 

    \item (\textbf{Sem divisores de zero}). Demonstre que um anel associativo comutativo unitário finito é um corpo se, e somente se, ele não possui divisores de zero. \textit{Dica: Exercício 9.1.} 

    \item (\textbf{Sequências de Cauchy}). Seja $F$ um corpo com um valor absoluto 
    \[
    |\cdot| : F \to \mathbb{R}_{\geq 0}
    \] 
    (isto é, para quaisquer $x,y \in F$ valem: $|x|=0 \iff x=0$; $|xy| = |x||y|$; $|x+y| \leq |x|+|y|$). 

    Mostre que as sequências $(x_n)_{n \geq 0}$ sobre $F$ tais que, para cada $\varepsilon > 0$, existe um correspondente $N > 0$ de modo que, para quaisquer $n,m \geq N$, vale $|x_n - x_m| < \varepsilon$, formam um anel com respeito às operações de adição e multiplicação termo a termo. 

    \item (\textbf{$R^X$}). Sejam $X$ um conjunto e $R$ um anel. Verifique que o conjunto $R^{(X)}$ das funções $f: X \to R$ tais que $\{x \in X : f(x) \neq 0\}$ é finito é: 
    \begin{enumerate}[label=(\roman*)]
        \item um grupo com respeito à soma pontual de funções; 
        \item um anel definindo-se o produto por 
        \[
        (f*g)(x) = \sum_{y \in X} f(y)g(y^{-1}x), \quad \text{se $X$ for um grupo};
        \] 
        \item um espaço vetorial sobre $R$ sob a operação escalar $(rf)(x) = r f(x)$, se $R$ for um corpo.
    \end{enumerate}

    \item (\textbf{Quaternions}). Seja $H$ o conjunto das matrizes complexas 
    \[
    q = \begin{pmatrix} a+id & -b-ic \\ b-ic & a-id \end{pmatrix}, \quad a,b,c,d \in \mathbb{R}.
    \] 
    Prove que a soma e o produto matriciais usuais tornam $H$ um anel de divisão (isto é, um ``corpo possivelmente não comutativo''). 

    Prove que 
    \[
    q = a1 + b i + c j + d k,
    \] 
    com 
    \[
    1 = \begin{pmatrix} 1 & 0 \\ 0 & 1 \end{pmatrix}, \quad 
    i = \begin{pmatrix} i & 0 \\ 0 & -i \end{pmatrix}, \quad 
    j = \begin{pmatrix} 0 & 1 \\ -1 & 0 \end{pmatrix}, \quad 
    k = \begin{pmatrix} 0 & i \\ i & 0 \end{pmatrix}.
    \] 

    Mostre que $i^2 = j^2 = k^2 = ijk = -1$, e $ij = k = -ji$. Prove ainda que 
    \[
    q\overline{q} = |q|^2 = a^2 + b^2 + c^2 + d^2 \geq 1.
    \]

\end{enumerate}

\end{document}
