\documentclass[a4paper,12pt]{article}
\usepackage[utf8]{inputenc}
\usepackage{amsmath,amssymb}
\usepackage[brazil]{babel}
\usepackage{geometry}
\geometry{margin=2.5cm}
\usepackage{enumitem} 
\usepackage{minipage-marginpar}

\pagestyle{empty}

\begin{document}

\begin{center}
    Álgebra 1 
\end{center}

\begin{center}
    \large Lista 03 \\
    \small (Divisão Euclidiana e M.D.C)
\end{center}

\begin{enumerate}[label=3.\arabic*.]
    \item (Sequência de Fibonacci). Prove que, na sequência $0,1,1,2,3,5,8,13,...$, na qual 
    cada termo a partir do terceiro é a soma dos dois precedentes, o m.d.c. de quaisquer dois 
    termos consecutivos é igual a 1.
    
    Seja $F_{n}$ e $F_{n+1}$ dois termos consecutivos quaisquer da sequência de Fibonacci.

    Sabemos, por definação, que : 
    \[
        F_{n+1} = F_{n} + F_{n-1}, n \geq 2 
    \]

    Pensando na sequência como uma divisão euclidiana de quociente igual à 1 temos,

    \[
        F_{n+1} = (1) \cdot F_{n} + F_{n-1}
    \]    

    Logo, o $mdc(F_{n+1},F_{n}) = mdc(F_{n},F_{n-1})$ pois se $a = b \cdot q + r$ então $mdc(a,b) = mdc(b,r)$

    Aplicando essa lógica redutiva até os primeiros termos da sequência temos : 

    $ \ldots = mdc(3,2) = mdc(2,1) = 1$

    \item ($GL_2(\mathbb{Z})$). Se $p, q, r$ e $s$ são inteiros tais que $ps - qr = \pm 1$, e 
    $a, b, a', b'$ são inteiros tais que 
    \begin{equation*}
        \left\{
            \begin{aligned}
                a' &= p a + q b,\\
                b' &= r a + s b
            \end{aligned}
        \right.
    \end{equation*}
    prove que $m.d.c.(a', b') = m.d.c.(a, b)$.
    
    A ideia é mostrar que $mdc(a,b) \mid mdc(a',b') $ pois a única maneira de inteiros positivos se dividirem mutualmente e
    se eles forem \textbf{iguais}

    \begin{enumerate}[label=\Roman*.]
        \item $mdc(a,b) = d$
        
        - $d \mid a \rightarrow a = d \cdot k_1 $ \\
        - $d \mid b \rightarrow b = d \cdot k_2 $
        
        Substituindo no sistema temos, 

        \begin{equation*}
            \left\{
                \begin{aligned}
                    a' &= p(dk_1) + q(dk_2) = d \cdot (pk_1 + qk_2) \\
                    b' &= r(dk_1) + s(dk_2) = d \cdot (rk_1 + sk_2)
                \end{aligned}
            \right.
        \end{equation*}

        Como $d \mid a' $ e $d \mid b' $, por definição, $d | mdc(a',b') = \boxed{mdc(a,b) | mdc(a',b')}$

        \item Pelo Sistema,
        
        $L_1 \cdot s - L_2 \cdot q \Rightarrow sa' - b'q = psa + qsb - qra - qsb$

        $ sa' - b'q = a \cdot (ps - qr), \text{ logo } a = \pm (a's - b'q)$

        De forma Analoga, 
        $b = \pm (pb' - ra')$

        Assim, seja $mdc(a',b') = d'$

        Como, $d' \mid a'$ e $d' \mid b'$, então $d'$ divide qualquer combinação linear entre eles

        Logo, $d' \mid a$ e $d' \mid b$ portanto $d' \mid mdc(a,b) = \boxed{mdc(a',b') \mid mdc(a,b)}$
    \end{enumerate}    

    \item (Para no m.d.c.). Sejam $a$ e $b$ dois inteiros $>0$. Faça $a_0 = a, a_1 = b$, e para $n \geq 1$ 
    defina $a_{n+1}$ por $a_{n-1} = a_n q_n + a_{n+1}$, com $0 \leq a_{n+1} < a_n$ desde que $a_n > 0$. 
    Demonstre que existe $N \geq 1$ tal que $a_{N+1} = 0$, e que $a_N = m.d.c.(a, b)$.

    $a_0 = 0, a_1 = b$

    $a_{n-1} = a_n q_n + a_{n+1}$, com $0 \leq a_{n+1} < a_n$

    \begin{enumerate}[label=\Roman*.]
        \item Existe $N \ge 1$ tal que $a_{N+1} = 0$
        
        Pela condição do algoritmo o resto das divisões é uma sequência de inteiros 
        decrescentes e não negativa

        Assim, pelo Princípio da Boa Ordenação em algum momento essa sequência 
        deve atingir o menor inteiro não negativo que é o $0$.
        
        \item $a_N = mdc(a,b)$
    
        Quando o resto é $0$ temos o seguinte : 

        $a_{N-1} = a_N \cdot q_N \rightarrow a_{N-1} \mid a_N $

        $a_{N-2} = a_{N-1} \cdot q_{N-1} + a_N$, como $a_N$ divide ambos os termos ele também divide $a_{N-2}$

        Aplicando o mesmo processo cheagamos que $a_N \mid a$ e $a_N \mid b$

        Além disso, seja $d$ um divisor comum qualquer de $a$ e $b$, temos que : 

        $a_0 = a_1 q_1 + a_2 \Leftrightarrow a_2 = a_0 - a_1 q_1$, como $d \mid a_0$ e $d \mid a_1q_1$,logo 
        $d$ divide a diferença entre eles.

        Descendo por todas as equações chegamos que $d$ divide todos os restos $a_2,a_3,\ldots,a_n$

        Se todo divisor comum de $a$ e $b$ também divide $a_n$, então $a_n \ge$ que qualquer outro divisor comum

        Portanto $mdc(a,b) = a_n$

    \end{enumerate}

    \item (M.d.c. e combinação linear). Usando a notação do exercício 3.3, mostre que $a_n$ 
    pode ser escrito na forma $ax + by$ com $x$ e $y$ inteiros, para cada $n \geq 0$ e $\leq N$.
    
    Seja $a = a_0$ e $b = a_1$ , $a_n = ax_n + by_n$ 
    
    para $n = 0$ temos :
    
    $a_0 = a = a(1) + b(0)$, onde $x_0 = 1, y_0 = 0$

    para $n = 1$ temos :
    
    $a_1 = b = a(0) + b(1)$, onde $x_0 = 0, y_0 = 1$

    Suponhamos que essa propriedade seja válida para cada $n$ com $0 \leq n \leq N$

    Daí,
    \begin{align*}
        a_{N-1} a_{N-1} \cdot q_{N-1} + a_N \rightarrow a_{N} &= a_{N-2} - a_{N-1} \cdot q_{N-1}\\
        a_N &= (a \cdot x_{N-2} + b \cdot y_{N-2}) - (a \cdot x_{N-1} + b \cdot y_{N-1}) \cdot q_{N-1} \\
        a_N &= a(x_{N-2} - x_{N-1} \cdot q_{N-1}) + b(y_{N-2} - y_{N-1} \cdot q_{N-1}) \\
        &= a(x_N) + b(y_N)
    \end{align*}

    Logo, pelo Princípio da Indução Matemática a afirmação é válida para cada $n$ com $0 \leq n \leq N$
    
    \item (Calculando m.d.c. I). Use o procedimento descrito nos Exercícios 3.3 e 3.4 para 
    encontrar $m.d.c.(a,b)$ e resolver $ax + by = m.d.c.(a,b)$ em cada um dos seguintes casos:
    \begin{itemize}
        \item[(i)] $a = 6188, b = 4709$
        
        \begin{minipage}[t]{0.4\linewidth}
            \begin{align*}
                6188 &= (1)\cdot 4709 + 1479 \\
                4709 &= (3)\cdot 1479 + 272 \\
                1479 &= (5)\cdot 272 + 119 \\
                272 &= (2)\cdot 119 + 34 \\
                119 &= (3)\cdot 34 + 17 \\
                34 &= (2)\cdot 17 + 0
            \end{align*}
        \end{minipage}\hfill
        \begin{minipage}[t]{0.6\linewidth}
            \begin{align*}
                17 &= 119 + (-3)\cdot 34 \\
                17 &= (-3)\cdot 272 + (7)\cdot 119 \\
                17 &= (7)\cdot 1479 + (-38)\cdot 272 \\
                17 &= (-38)\cdot 4709 + (121)\cdot 1479 \\
                17 &= (121)\cdot 6188 + (-159)\cdot 4709
            \end{align*}
        \end{minipage}

        $\boxed{x = 121, y = -159}$

        \item[(ii)] $a = 81719, b = 52003$
        
        \begin{minipage}[t]{0.4\linewidth}
            \begin{align*}
                81719 &= (1)\cdot 52003 + 29716 \\
                52003 &= (1)\cdot 29716 + 22287 \\
                29716 &= (1)\cdot 22287 + 7429 \\
                22287 &= (3)\cdot 7429 + 0
            \end{align*}
        \end{minipage}\hfill
        \begin{minipage}[t]{0.6\linewidth}
            \begin{align*}
                7429 &= (-1) \cdot 22287 + (1)\cdot 29716 \\
                7429 &= (-1)\cdot 52003 + (2)\cdot 29716 \\
                7429 &= (2)\cdot 29716 + (-3)\cdot 52003
            \end{align*}
        \end{minipage}
        
        $\boxed{x = 2, y = -3}$

        \item[(iii)] $a = 33649, b = 30107$
    
        \begin{minipage}[t]{0.4\linewidth}
            \begin{align*}
                33649 &= (1)\cdot 30107 + 3542 \\
                30107 &= (8)\cdot 3542 + 1771 \\
                3542 &= (2)\cdot 1771 + 0
            \end{align*}
        \end{minipage}\hfill
        \begin{minipage}[t]{0.6\linewidth}
            \begin{align*}
                1771 &= 30107 + (-8)\cdot 3542 \\
                1771 &= (-8) \cdot 33649 + (9)\cdot 30107
            \end{align*}
        \end{minipage}
    
        $\boxed{x = 8, y = -9}$

    \end{itemize}
    
    \item (Homogeneidade). Se $a,b,\ldots,c$ e $m$ são inteiros e $m > 0$, mostre que :
    
    $m.d.c.(ma,mb,\ldots,mc) = m \cdot m.d.c.(a,b,\ldots,c)$.
    
    Seja $mdc(a,b,\ldots,c) = ax + by + \ldots + cz$
    
    Neste caso,

    $mdc(ma,mb,\ldots,mc) = max + mby + \ldots + mcz$
    
    $= m \cdot (ax + by + \ldots cz)$
    
    $= m \cdot mdc(a,b,\ldots,c)$      

    \item (Representação canônica de números racionais). Prove que cada número racional 
    pode ser escrito de maneira única como $\frac{m}{n}$ com $m.d.c.(m,n) = 1$ e $n>0$.

    Suponha que um número racional $q$ tenha duas representações

    $q = \frac{m_1}{n_1} = \frac{m_2}{n_2}$, onde $mdc(m_1,n_1) = 1, n_1 > 0$ e $mdc(m_2,n_2) = 1, n_2 > 0$

    Assim, $m_1n_2 = m_2n_1$

    Como o $mdc(m_1,n_1) = 1$ então $n_1$ deve dividir $n_2$ e de forma análoga $n_2$ deve dividir $n_1$

    Logo, $n_1 = n_2$ pois $n_1 \mid n_2$ e $n_2 \mid n_1$
    
    $n_1 = n_2 \Rightarrow m_1 = m_2$

    Como $m_1 = m_2$ e $n_1 e n_2$ as representações são identicas e únicas. 

    \item (Mínimo múltiplo comum). Dados $a$ e $b$ dois inteiros positivos não-nulos, mostre 
    que na igualdade $ab = d m$ vale que $d$ é o m.d.c. de $a$ e $b$ se, e somente se, $m$ é o 
    m.m.c. de $a$ e $b$ (i.e., $m$ é um inteiro $\geq 0$, $a$ divide $m$ e $b$ divide $m$, e para cada 
    inteiro $m' \geq 0$ tal que $a$ divide $m'$ e $b$ divide $m'$, vale que $m$ divide $m'$).

\end{enumerate}

\end{document}
