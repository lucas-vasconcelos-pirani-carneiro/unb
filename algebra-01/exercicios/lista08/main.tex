\documentclass[a4paper,12pt]{article}
\usepackage[utf8]{inputenc}
\usepackage{amsmath,amssymb}
\usepackage[brazil]{babel}
\usepackage{geometry}
\geometry{margin=2.5cm}
\usepackage{enumitem} 

\pagestyle{empty}

\begin{document}

\begin{center}
    \large Álgebra 1
\end{center}

\begin{center}
    \large Lista 08 \\
    \small (Grupos)
\end{center}
  
\begin{enumerate}[label=8.\arabic*.]

    \item (\textbf{Permutações I}). Mostre que o conjunto $S_3$ de todas as permutações (i.e., bijeções) 
    $f: \{1,2,3\} \to \{1,2,3\}$ é um grupo com respeito à operação de composição de funções (i.e., para quaisquer $f$ e $g$ em $S_3$ e $j$ em $\{1,2,3\}$, tem-se $(gf)(j) = g(f(j))$). 

    Escrevendo uma tal permutação na forma 
    \[
    \begin{pmatrix} 
        1 & 2 & 3 \\ 
        f(1) & f(2) & f(3) 
    \end{pmatrix},
    \] 
    note que 
    \[
    S_3 = \left\{ 
    \begin{pmatrix}1&2&3\\1&2&3\end{pmatrix}, 
    \begin{pmatrix}1&2&3\\1&3&2\end{pmatrix}, 
    \begin{pmatrix}1&2&3\\2&1&3\end{pmatrix}, 
    \begin{pmatrix}1&2&3\\2&3&1\end{pmatrix}, 
    \begin{pmatrix}1&2&3\\3&1&2\end{pmatrix}, 
    \begin{pmatrix}1&2&3\\3&2&1\end{pmatrix} 
    \right\}.
    \] 

    Verifique que o elemento inverso da permutação 
    \(
    \begin{pmatrix}1&2&3\\2&3&1\end{pmatrix}
    \) 
    é 
    \(
    \begin{pmatrix}1&2&3\\3&1&2\end{pmatrix}
    \) 
    e calcule os produtos 
    \[
    \begin{pmatrix}1&2&3\\3&2&1\end{pmatrix} 
    \begin{pmatrix}1&2&3\\2&3&1\end{pmatrix} 
    \quad \text{e} \quad 
    \begin{pmatrix}1&2&3\\2&3&1\end{pmatrix} 
    \begin{pmatrix}1&2&3\\1&3&2\end{pmatrix}.
    \]

    \item (\textbf{$SL_2(\mathbb{Z})$}). Prove que as matrizes 
    \(
    \begin{pmatrix}a & b \\ c & d \end{pmatrix}
    \), com entradas em $\mathbb{Z}$, tais que $ad - bc = 1$, formam um grupo não comutativo com respeito ao produto usual de matrizes. 

    \item (\textbf{Produto de grupos I}). Se $G$ e $H$ são grupos, verifique que o conjunto 
    $G \times H$ consistindo em todos os pares $(g,h)$ em que $g$ é um elemento de $G$ e $h$ é um elemento de $H$ 
    torna-se um grupo com a operação 
    \[
    (g_1,h_1)(g_2,h_2) = (g_1g_2, h_1h_2),
    \] 
    onde o produto $g_1g_2$ é tomado em $G$ e o produto $h_1h_2$ é tomado em $H$.

    \item (\textbf{Um grupo de transformações}). Se $x = (x_1, x_2) \in \mathbb{R} \times \mathbb{R}$, em que $\mathbb{R}$ é o grupo aditivo dos números reais, considere $|x| = \sqrt{x_1^2 + x_2^2}$. Demonstre que as funções $f : \mathbb{R}^2 \to \mathbb{R}^2$ tais que $f(0) = 0$ e $|f(x) - f(y)| = |x - y|$, para quaisquer $x$ e $y$, formam um grupo sob a operação de composição de funções. 

    \item (\textbf{Um grupo poliedral}). Verifique que um cubo possui as seguintes simetrias (bijeções do cubo em si mesmo que preservam a distância): a identidade; uma rotação $\chi$ de $120^\circ$ em torno do eixo através de um dos quatro pares de vértices opostos; uma rotação $\iota$ de $90^\circ$ em torno do eixo através do meio de um dos seis pares de arestas opostas; a rotação $\upsilon$ de $180^\circ$ em torno do eixo através do meio de um dos três pares de faces opostas; e, a reflexão pelo plano paralelo a, e no meio de, duas faces opostas. 

    Verifique que todas as simetrias de rotação do cubo realizam todas as permutações das 4 diagonais maiores dele, e que as 24 restantes simetrias do cubo combinam uma reflexão e uma rotação. 

    \item (\textbf{Grafos II}). Dizemos que dois caminhos $p$ e $p'$ em um grafo $\Gamma$ são homotópicos se existe uma sequência finita $p = p_1, \ldots, p_t = p'$ na qual cada termo é obtido do anterior por um único processo de inserção ou eliminação de pares adjacentes do tipo $ee^{-1}$ com $e \in E(\Gamma)$. Verifique que isso define uma relação de equivalência. 

    Fixado $v \in V(\Gamma)$, se $p$ e $p'$ são caminhos de $v$ em $v$, defina o produto $pp'$ por concatenação. Imitando a definição do grupo aditivo $\mathbb{Z}/m\mathbb{Z}$ a partir de $\mathbb{Z}$, mostre que o conjunto das classes de equivalência de caminhos de $v$ em $v$ é um grupo no qual a classe representada pelo caminho vazio é o elemento identidade. 

\end{enumerate}

\end{document}
