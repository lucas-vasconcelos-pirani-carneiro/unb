\section*{Divisão Euclidiana e Máximo Divisor Comum}

Vamos começar com um exemplo para calcular o $mdc$ entre dois \textbf{números inteiros}.

\begin{exemplo}[: {$mdc(2014,1486) = 106$}]
\[
\begin{aligned}
    2014 &= 1 \cdot 1484 + 530 \\
    1484 &= 2 \cdot 530 + 424 \\
    530 &= 1 \cdot 424 + 106 \\
    424 &= 4 \cdot 106 + 0.
\end{aligned}
\]
\end{exemplo}

\noindent Analisando de \underline{baixo para cima}, $106$ divide $1484$ ($106|1484$) e $106$ divide $2014$ 
($106|2014$), e de \underline{cima para baixo}, se um \textbf{inteiro} $d$ divide $2014$ e $1484$, então $d$ divide $106$ ($d|106$).

\vspace{0.05cm}
\noindent Assim, podemos reescrever a equações de cima para baixo do seguinte modo : 

\begin{align*}
    530 &= 2014 + (-1) \cdot 1484 \\
    424 &= 1484 + (-2) \cdot 530 = (-2) \cdot 2014 + 3 \cdot 1484 \\
    106 &= 530 + (-1) \cdot 424 = 3 \cdot 2014 + (-4) \cdot 1484. \\
\end{align*}

\subsection*{Divisão Euclidiana}

\noindent\underline{\underline{\textbf{($\ast$) Lema 4.1. }}} : \textbf{\textcolor{teal}{Divisão Euclidiana}}

Se $a$ e $d$ são \textbf{inteiros} com $d > 0$, então existe um \underline{único maior múltiplo} $qd$ de $d$ que é 
\underline{menor do que ou igual} a $a$; ele pode ser caracterizado por $qd \leq a < (q + 1)d$, ou por : 
\begin{center}
    $a = qd + r \text{, com } 0 \leq r < d$
\end{center}

\noindent $r$ : \textbf{Resto} da divisão de $a$. \\
$d$ : \textbf{Divisor}. \\
$q$ : \textbf{Quociente}.

\vspace{0.2cm}
\noindent\textit{Demonstração : }

O conjunto dos inteiros positivos da forma $a - zd \text{ , com } z \in \mathbb{Z}$ é não-vazio, pois podemos
tomar $z = -N$ em que $N$ é um inteiro não-negativo suficientemente grande.

Assim, pelo \textbf{\textcolor{teal}{Princípio da Boa Ordenação}}, digamos que seja r o menor elemento daquele conjunto, 
e escrevamos $r = a - qd$. Então, $r \ge 0$ e, $r < d$ pois caso contrário $a - (q + 1)d$ pertenceria ao conjunto e seria $< r$.

\begin{exemplo}[]
    Os \textbf{números ímpares} são da forma $2n + 1 \text{,com } n \in \mathbb{Z}$. \\
    A \underline{diferença} entre quaisquer dois \textbf{números ímpares} é um \textbf{número par}.
\end{exemplo}

\noindent\underline{\underline{\textbf{($\ast$) Proposição 4.2. }}} : Seja $M$ um conjunto não-vazio de inteiros. 
Se $M$ é \textbf{fechado} com respeito à \underline{subtração}, então existe um único $m \ge 0$ tal que M é o conjunto
de \textbf{todos} os \underline{múltiplos de m} :

\begin{center}
    $ M = \{mz : z \in \mathbb{Z} \} = m\mathbb{Z} $
\end{center}

\vspace{0.2cm}
\noindent\textit{Demonstração : }

Começemos com algumas observações. Se $x \in M$, então pela hipótese $0 = x - x \in M$, e
$-x = 0 - x \in M$. Se, além disso, $y \in M$, então $y + x = y - (-x) \in M$, logo $M$ é fechado com respeito à
adição. Se $x \in M$ e $nx \in M$ em que $n$ é um inteiro não-negativo qualquer, então $(n + 1)x \in M$. Portanto,
pelo \textbf{\textcolor{teal}{Princípio da Indução Matemática}}, $nx \in M$ para cada $n \ge 0$, e logo para
cada $n \in \mathbb{Z}$. Finalmente, todas as combinações lineares de elementos de $M$ com coeficientes inteiros ainda
pertencem a $M$. Como essa propriedade resulta em $M$ ser fechado com respeito à adição e à subtração,
ela é equivalente à hipótese sobre $M$.  

Se $M = \{0\}$, a proposição é verdadeira tomando-se $m = 0$. Caso contrário, o conjunto dos elementos
$> 0$ em $M$ é não-vazio. Tomemos $m$ como o menor desses elementos. Todos os múltiplos de $m$ pertencem
a $M$. Para cada $x \in M$, aplicamos o Lema 4.1 (Divisão Euclidiana) e escrevemos $x = my + r$ com
$0 \le r < m$; então, $r = x - my \in M$. Pela definição de $m$, isto implica $r = 0$, ou seja, $x = my$. Portanto,
$M = m\mathbb{Z}$. Finalmente, como $m$ é o menor elemento $> 0$ em $m\mathbb{Z}$, ele é unicamente determinado quando
$M$ é dado.

\vspace{0.2cm}
\noindent\underline{\underline{\textbf{($\ast$) Corolário 4.3. }}} : Se $a, b, \ldots, c$ são inteiros em qualquer quantidade finita, 
então existe um único inteiro $d \ge 0$ tal que o conjunto de todas as combinações lineares $ax + by + \cdots + cz$ de 
$a, b, \ldots, c$ com coeficientes inteiros $x, y, \ldots, z$ consiste em todos os múltiplos de $d$.

\begin{center}
    $ \{ax + by + \ldots + cz : x,y,\ldots,z \in \mathbb{Z} \} = d\mathbb{Z} $
\end{center}

\vspace{0.2cm}
\noindent\textit{Demonstração : }

Aplique a Proposição ao conjunto das tais combinações lineares.

\vspace{0.2cm}
\noindent\underline{\underline{\textbf{($\ast$) Corolário 4.4. }}} : Sob as notações e hipóteses do Corolário 4.3, vale que $d$ é um 
divisor de cada um dos inteiros $a, b, \ldots, c$ e cada divisor comum desses inteiros é um divisor de $d$.

\vspace{0.2cm}
\noindent\textit{Demonstração : }

Cada um dos inteiros $a, b, \ldots, c$ pertence ao conjunto das suas combinações lineares; e cada divisor comum de 
$a, b, \ldots, c$ é um divisor de cada uma das suas combinações lineares, e em particular, de $d$.

\subsection*{Máximo Divisor Comum}

\noindent\underline{\underline{\textbf{($\ast$) Definição 4.5. }}} : O inteiro $d$ definido nos corolários da Proposição é chamado 
de o \textbf{máximo divisor comum} (ou abreviadamente \textbf{$m.d.c.$}) de $a, b, \ldots, c$; ele é denotado por $(a, b, \ldots, c)$.  

Como o $m.d.c.$ $(a, b, \ldots, c)$ pertence ao conjunto das combinações lineares, 
ele pode ser escrito da forma:
\[
(a, b, \ldots, c) = ax_0 + by_0 + \ldots + cz_0,
\]
em que $x_0, y_0, \ldots, z_0$ são inteiros.

\begin{exemplo}
    \begin{enumerate}
        \item Temos $(6, 10, 15) = 1$, $(6, 10) = 2$, $(6, 15) = 3$, e $(10, 15) = 5$.
        \item Para $a \ge 0$, temos $(a, b) = a$ se, e somente se, $a \mid b$.
        \item Se $a = qb + c$, então $(a, b) = (b, c)$.
    \end{enumerate}
\end{exemplo}
