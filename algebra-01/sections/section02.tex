\section{Fórmula Binomial}

    \begin{definicao}[: Fórmula Binomial]
        \begin{itemize}[left=0.5cm, align=left, nosep]
            \item Seja $n,k \in \mathbb{Z} \ge 0$. Definimos :
        \end{itemize}
        
        \begin{equation*}
            \binom{n}{k} = \begin{cases}\displaystyle
                \frac{n \cdot (n-1) \cdot \dotsc \cdot (n-k-1) }{k!}, & \text{se } n \ge k,\\[6pt] 
                0, & \text{se } n < k,
            \end{cases}        
        \end{equation*}

        \hspace{0.5cm} em que $k!$ denota o \textbf{produto} dos inteiros $\ge 1$ e $\leq k$.

    \end{definicao}
    
    \begin{itemize}[left=0.5cm, align=left, nosep]
        \item Dessa definição decorre que : 
    \end{itemize}

    \vspace{-0.75cm}
    \begin{align*}
        \binom{n}{k-1} + \binom{n}{k} = \binom{n + 1}{k}, \text{ sempre que $1 \leq k \leq n$.}
    \end{align*}

    \underline{\underline{ \textbf{\textcolor{cinzaEscuro}{($\ast$) Afirmação}} }} : Dado um inteiro $n \ge 1$ para cada inteiro $k$
    com $0 \leq k \leq n$, vale que o \underline{máximo} $\binom{n}{k}$ é um inteiro.

    \textit{Demonstração : }

    De fato, para $n=1$ temos :
    \begin{align*}
        \binom{1}{0} = 1 = \binom{1}{1}
    \end{align*}

    Suponhamos que essa afirmação seja válida para cada n com $0 \leq n \leq N+1$, sempre que $0 \leq k \leq n$.

    Daí,
    $\binom{N+1}{0} = 1 = \binom{N+1}{N+1}$, e se $1 \leq k < N$ sabemos que $\binom{N+1}{N+1}$ é a soma de dois inteiros.

    Logo, pelo Princípio da Indução Matemática a afirmação é válida.

    \underline{\underline{ \textbf{\textcolor{cinzaEscuro}{($\ast$) Proposição}} }} : Dado dos números reais (ou complexos) $a$ e $b$ e
    um inteiro $n \ge 1$ vale a seguinte afirmação : 

    \vspace{-1cm}
    \begin{align*}
        (a+b)^n &= a^n + \dotsc + \binom{n}{k}a^{n-k} \cdot b^{k} + \dotsc b^n\\
        &= \sum_{0 \leq k \leq n}^{} \binom{n}{k}a^{n-k} \cdot b^{k}\\
    \end{align*}

    \vspace{2.1cm}
    \textit{Demonstração : }

    Para $n=1$, temos :
    \begin{align*}
        (a+b)^1 = \binom{1}{0}a + \binom{1}{1}b 
    \end{align*}

    Suponhamos que a fórmula binomial seja verdadeira para cada $n$ com $1 \leq n \leq N+1$

    Daí,
    \begin{align*}
        (a+b)^{N+1} &= (a+b)^{N} \cdot (a+b) \\ 
        &= a^{N+1} + \binom{N}{1}a^{N}b + \binom{N}{2}a^{N-1}b^2 + \dotsc + \binom{N}{N}ab^{N} \\
        &+ \binom{N}{0}a^{N}b + \binom{N}{1}a^{N-1}b^2 + \dotsc + b^{N+1} \\
        &= a^{N+1} + \binom{N+1}{1}a^{N}b + \binom{N+2}{2}a^{N-1}b^2 + \dotsc + b^{N+1}
    \end{align*}

    Logo, pelo Princípio da Indução Matemática a fórmula binomial está demostrada para cada $n \ge 1$.

    \vspace{0.5cm}
    
    \begin{exemplo}[]
        Mostre que para cada inteiro $n \ge 0$ vale que $n^5 - n$ é um \underline{múltiplo de 5}.
    \end{exemplo}
    
    De fato, para $n = 0$ temos
    \begin{align*}
        0^5 - 0 = 5 \cdot (0)
    \end{align*}

    Suponhamos que $n^5 - n$ é um multíplo de 5 para cada $n$ com $0 \leq n < N+1$

    Daí,
    \begin{align*}
        (N+1)^5 - (N+1) &= N^5 - N + 5N^4 + 10N^3 + 10N^2 + 5N + 1 - 1 \\
        &= N^5 - N + 5 \cdot (N^4 + 2N^3 + 2N^2 + N) \\
        \text{que é um multíplo de 5}
    \end{align*}

    Logo, pelo Princípio da Indução Matemática $n^5 - n$ é um multíplo de 5 para $n \ge 0$. 

    \vspace{2cm}
