\section*{Inteiros Gaussianos}

Recordemos o conceito de um \textbf{número complexo}: este é um número da forma 
$a = x + iy$, onde $x$ e $y$ são números reais e $i$ satisfaz $i^2 = -1$. 
As regras para a \underline{adição} e a \underline{multiplicação} são as usuais:
\[
    (x + iy) + (x' + iy') = (x + x') + i(y + y'),
\]
\[
    (x + iy)(x' + iy') = (xx' - yy') + i(yx' + xy').
\]

Provido de tais operações, o conjunto $\mathbb{C}$ dos \textbf{números complexos} torna-se um \textbf{anel}
\underline{associativo, comutativo e unitário}, com $1 = 1 + i \cdot 0$.  

Se $a = x + iy$, escrevemos $\boxed{\overline{a} = x - iy}$ e chamamos $\overline{a}$ de 
\textbf{conjugado complexo} de $a$; o conjugado de $\overline{a}$ é $a$.  

A aplicação $a \mapsto \overline{a}$ é uma \underline{bijeção} de $\mathbb{C}$ sobre \textbf{si mesma} que 
\underline{preserva} as operações de adição e multiplicação; logo, ela é um \underline{\textit{automorfismo}} de 
$\mathbb{C}$, isto é, um \underline{\textit{isomorfismo}} de $\mathbb{C}$ em $\mathbb{C}$.

Escrevemos $N(a) = a \cdot \overline{a}$ e chamamos $N(a)$ de \underline{\textit{norma}} de $a$.  
Pela regra da multiplicação, se $a = x + iy$ então : 
\[
N(a) = x^2 + y^2,
\]
e, pela \textbf{comutatividade} da multiplicação, temos
\[
N(ab) = N(a)N(b).
\]

A norma de $a$ é $0$ se, e somente se, $a = 0$; caso contrário, ela é um número real $> 0$.  
Consequentemente, para cada $a = x + iy \neq 0$, consideramos
\[
a' = N(a)^{-1}\,\overline{a} = \frac{x}{N(a)} - i\frac{y}{N(a)}.
\]

\noindent Então $aa' = 1$, e, para cada $b \in \mathbb{C}$, vale $a(a'b) = b$.  
Reciprocamente, se $az = b$, então $a'(az) = a'b$, donde, pela \underline{associatividade}, obtemos 
$z = a'b$. Isso mostra que $\mathbb{C}$ é um \textbf{corpo}.

De maneira usual, fazemos corresponder o número complexo 
$a = x + iy$ ao ponto $(x,y)$ no plano euclidiano; sua distância da origem $0$ é
\[
|a| = \sqrt{x^2 + y^2} = \sqrt{N(a)}.
\]
Esse valor é também chamado de \underline{\textit{valor absoluto}} de $a$.

\vspace{0.2cm}
\noindent Para nossos propósitos, vamos considerar, em vez de $\mathbb{C}$, o subconjunto 
\[
\mathbb{Z}[i] = \{x + iy \mid x, y \in \mathbb{Z}\}
\]
consistindo nos números complexos cujas \underline{partes real} e \underline{imaginária} são inteiras.  

Pode ser imediatamente verificado que $\mathbb{Z}[i]$ é um \textbf{anel} \underline{associativo, comutativo e unitário}, com 
$1 = 1 + i \cdot 0$. Ele é chamado \textbf{anel Gaussiano}, e seus elementos são chamados 
\textbf{inteiros Gaussianos}.  

A bijeção $a \mapsto \overline{a}$ preserva as operações desse anel.  
Se $a$ é um inteiro Gaussiano qualquer, então :
\[
N(a) = a \cdot \overline{a}
\]
é um inteiro $\geq 0$.  

Ocasionalmente também consideramos os números $x + iy$ com $x, y \in \mathbb{Q}$; como anteriormente, 
eles formam um \textbf{corpo} (o \textbf{corpo Gaussiano}).  

\subsection*{Divisibilidade e Associados}

Se $a, b, c \in \mathbb{Z}[i]$ e $b = ac$ , então : 

- $b$ é um \underline{múltiplo} de $a$, 

- $a$ \underline{divide} $b$ (ou que $a$ é um \underline{divisor} de $b$).  

\noindent Quando isso ocorre, $N(a)$ \underline{divide} $N(b)$.  

\noindent Cada inteiro Gaussiano \underline{divide} a sua \textbf{norma}.  

\begin{itemize}
    \item Um divisor de $1$ é chamado \textbf{invertível}.  
    \item Se $a = x + iy$ é \textbf{invertível}, então $N(a) = 1$. Como $x, y \in \mathbb{Z}$, isso implica que $a \in \{\pm 1, \pm i\}$.  
    \item Dois inteiros Gaussianos não nulos $a$ e $b$ \underline{dividem um ao outro} se, e somente se, diferem por um fator invertível, isto é, $a = ub$ com $u \in \{\pm 1, \pm i\}$. Nesse caso, chamamos $a$ e $b$ de \textbf{associados}.  
\end{itemize}

Dentre os \textbf{quatro associados} de um dado inteiro Gaussiano $b$, existe um único, digamos $a = x + iy$, satisfazendo $x > 0$ e $y \geq 0$. Este será chamado de \textbf{normalizado}.  

\begin{exemplo}
Dentre os associados $\pm 1 \pm i$ de $1+i$, apenas $\boxed{1+i}$ é normalizado.      
\end{exemplo}

Geometricamente, os \underline{pontos no plano} correspondentes aos associados de $b$ são obtidos 
a partir de $b$ por uma \underline{rotação} em torno de $0$ por um ângulo $n\pi/2$, com $n = 0,1,2,3$. 

O normalizado é aquele no \textbf{primeiro quadrante} (ou sobre o semieixo real positivo).  

\subsection*{Primos Gaussianos}

Um inteiro Gaussiano de norma $> 1$ é chamado \textbf{primo Gaussiano} se seus divisores são 
exatamente os seus \underline{associados} e os \underline{invertíveis}.  

\noindent Isso equivale a dizer que $q$ é um primo Gaussiano se:

- $q \neq 0$;

- $q$ \textbf{não} é invertível;

- $q$ \textbf{não possui divisor} com \underline{norma} $>1$ e $< N(q)$.

\noindent \textbf{Inteiros Ordinários} que são primos no sentido \underline{usual} serão chamados \textbf{primos racionais}.  

- Se $q \in \mathbb{Z}[i]$ e $N(q)$ é um \underline{primo racional}, então $q$ é um \underline{primo Gaussiano}.  

- A recíproca, porém, não é verdadeira. Exemplo, $N(3) = 9$, e $3$ é primo Gaussiano.  

- Os \textbf{associados} de um primo Gaussiano também são \underline{primos Gaussianos}, e existe apenas um normalizado.  

- Se $q$ é um \underline{primo Gaussiano}, então $\overline{q}$ também o é.  

Se $a \in \mathbb{Z}[i]$, $a \neq 0$ e não invertível, então \underline{todo divisor} de $a$ com norma mínima $> 1$ deve ser um \textbf{primo Gaussiano}.  

\vspace{0.3cm}
\noindent\underline{\underline{\textbf{($\ast$) Lema 7.1. }}} : \textbf{\textcolor{teal}{Divisão Euclidiana}}

Se $a, b \in \mathbb{Z}[i]$ com $b \neq 0$, então existe $q \in \mathbb{Z}[i]$ tal que
\[
N(a - bq) \leq \tfrac{1}{2}N(b).
\]

\noindent\textit{Demonstração : }

Para cada número real $t$, existe um maior inteiro $m \leq t$ tal que $m \leq t < m+1$.  
Chamamos $m'$ de inteiro mais próximo de $t$, isto é, $m$ ou $m+1$, de acordo com se $t-m \leq m+1-t$ ou não.  
Assim, $|t - m'| \leq \tfrac{1}{2}$.  

Seja $z = x + iy \in \mathbb{C}$. Tomando $m$ e $n$ inteiros mais próximos de $x$ e $y$, respectivamente, e definindo $q = m + in$, obtemos
\[
N(z - q) = (x-m)^2 + (y-n)^2 \leq \tfrac{1}{4} + \tfrac{1}{4} = \tfrac{1}{2}.
\]

Aplicando isso para $z = \tfrac{a}{b}$, com $a,b \in \mathbb{Z}[i]$, segue que $q$ satisfaz a propriedade desejada.

\begin{exemplo}
    \begin{enumerate}
        \item Para $a = 11 + 10i$ e $b = 4+i$:
        \[
            \frac{a}{b} = \frac{(11+10i)(4-i)}{N(4+i)} = \frac{54+29i}{17} \approx 3{,}17 + 1{,}71i.
        \]
        Tomando $q = 3 + 2i$, obtemos $a - bq = 1 - i$, e
        \[
            N(a-bq) \leq \tfrac{1}{2}N(b).
        \]

        \item Para $a = 3i$ e $b = 1+i$, no lema, há 4 possíveis escolhas de $q$, e verifica-se que a desigualdade não pode ser melhorada em geral.
    \end{enumerate}
\end{exemplo}


\begin{questao}[\textbf{(Fossas Gaussianas).}  ]
    Existe uma sequência infinita de primos Gaussianos distintos tal que haja uma 
    limitação para as diferenças entre os números consecutivos da sequência?
\end{questao}
