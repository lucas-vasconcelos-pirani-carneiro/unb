\section{Provas}

    \subsection{Prova 1}
        
        \subsubsection*{Questão 1}
            Seja $p$ primo e $n$ um número inteiro.  
            \begin{enumerate}[label=\alph*), left=0.5cm, align=left, nosep]
                \item Mostre por indução que $p$ é múltiplo de $n^p - n$ para $n \geq 0$.  
                \item Mostrar que o resultado do item anterior vale para $n \in \mathbb{Z}$ (incluindo negativos).
            \end{enumerate}
        
        \subsubsection*{Questão 2}
            Dado o seguinte $X^7 - 3X^5 + 2X^4 \text{ e } X^5 + X^4 - 2X^3 - X^2 - X + 2$
            \begin{enumerate}[label=\alph*), left=0.5cm, align=left, nosep]
                \item Encontre o m.d.c entre os polinômios
                \item Escrever o polinômio (provavelmente o m.d.c ou um deles) na forma fatorada pelas raízes
                \item Encontrar uma combinação linear que é igual à ...
            \end{enumerate}
        
        \subsubsection*{Questão 3}
            \begin{enumerate}[label=\alph*), left=0.5cm, align=left, nosep]
                \item Mostre que existem infinitos números primos usando o argumento de euclides 
                \item Mostre que existem infinitos números primos gaussianos
            \end{enumerate}

        \subsubsection*{Questão 4}
            
            Dada a seguinte operação ... descriavia a operação x1,x2... com y1,y2... e z1,z2 ...    

            \begin{enumerate}[label=\alph*), left=0.5cm, align=left, nosep]
                \item A operação é associativa ? Justifique  
                \item A operação é comutativa ? Justifique
            \end{enumerate}

    \subsection{Prova 2}
        \subsubsection*{Questão 1}

        Sejam $G$ e $G'$ grupos e seja $f: G \to G'$ uma função bijetiva tal que $f(xy) = f(x)f(y)$ para todos $x, y \in G$.

        \begin{enumerate}[label=\alph*), left=0.5cm, align=left, nosep]
            \item Considere o conjunto $S = \{ x \in G\ |\ f(x) = e' \}$, onde $e'$ é a identidade de $G'$. Verifique se $S$ é um subgrupo de $G$.
            \item Seja $H$ um subconjunto de $G$ definido por alguma propriedade envolvendo multiplicação. Analise as condições para que $H$ seja um subgrupo de $G$.
        \end{enumerate}

        \subsubsection*{Questão 2}

        Seja $G$ um grupo comutativo (abeliano). Verifique que:
        \begin{enumerate}
            \item [(i)] Se $n$ é um inteiro qualquer, então $H = \{ x^n\ |\ x \in G \}$ é um subgrupo de $G$.
            \item [(ii)] Se $H$ e $K$ são subgrupos de $G$, então $S = \{ hk\ |\ h \in H,\, k \in K \}$ é um subgrupo de $G$.
        \end{enumerate}

        \subsubsection*{Questão 3}

        Considere os conjuntos $(\mathbb{Z}/32\mathbb{Z})^\times$ e $(\mathbb{Z}/34\mathbb{Z})^\times$, isto é, os grupos multiplicativos dos elementos invertíveis em $\mathbb{Z}/32\mathbb{Z}$ e $\mathbb{Z}/34\mathbb{Z}$, respectivamente.

        \begin{enumerate}[label=\alph*), left=0.5cm, align=left, nosep]
            \item Existe um isomorfismo de grupos entre $(\mathbb{Z}/32\mathbb{Z})^\times$ e $(\mathbb{Z}/34\mathbb{Z})^\times$? Justifique sua resposta.
            \item Os anéis $\mathbb{Z}/32\mathbb{Z}$ e $\mathbb{Z}/34\mathbb{Z}$ são isomorfos? Existe um isomorfismo entre eles? Justifique sua resposta.
        \end{enumerate}

        \subsubsection*{Questão 4}

        \begin{enumerate}[label=\alph*), left=0.5cm, align=left, nosep]
            \item Para um inteiro $m > 1$, determine se $\mathbb{Z}/m\mathbb{Z}$ é um corpo. E determine se o grupo multiplicativo $(\mathbb{Z}/m\mathbb{Z})^{\times}$ é um grupo cíclico.
            \item Para $p$ primo, determine se $\mathbb{Z}/p\mathbb{Z}$ é um corpo. E determine se o grupo multiplicativo $(\mathbb{Z}/p\mathbb{Z})^{\times}$ é cíclico.
        \end{enumerate}

        \subsubsection*{Questão 5}

        \begin{enumerate}[label=\alph*), left=0.5cm, align=left, nosep]
            \item Seja $A = \{\, x + y \sqrt{2} \mid x, y \in \mathbb{Z} \,\}$. Prove que $A$ é um anel, com as operações usuais.
            \item Seja $K = \{\, x + y \sqrt{2} \mid x, y \in \mathbb{Q} \,\}$. Prove que $K$ é um corpo, com as operações usuais.
        \end{enumerate}

        \subsubsection*{Questão 6}

        Considere o conjunto $X = \{1, 2, 3, 4, 5\}$ e as funções (permutações) $\sigma$ e $\tau$ dadas por:
        \[
        \sigma(1) = 2,\ \sigma(2) = 3,\ \sigma(3) = 4,\ \sigma(4) = 5,\ \sigma(5) = 1
        \]
        \[
        \tau(1) = 1,\ \tau(2) = 5,\ \tau(3) = 4,\ \tau(4) = 3,\ \tau(5) = 2
        \]

        \begin{enumerate}[label=\alph*), left=0.5cm, align=left, nosep]
            \item Analise a bijetividade das funções $\sigma$ e $\tau$ e, se possível, descreva a ordem de $\sigma$ por composições sucessivas ($\sigma^k$).
            \item Verifique se $\sigma^4 \circ \tau = \sigma \circ \tau$. Justifique detalhadamente.
        \end{enumerate}

        Além dos itens já listados, as perguntas sobre permutações podem incluir as seguintes hipóteses frequentes:

        \begin{itemize}
            \item Determinar explicitamente o mapeamento composto: calcular e escrever $\sigma \circ \tau(x)$ para cada $x \in X$.
            \item Reescrever $\sigma \circ \tau$ em notação de ciclo ou como produto de ciclos.
            \item Identificar a ordem da permutação $\sigma \circ \tau$, isto é, o menor $n$ tal que $(\sigma \circ \tau)^n$ seja a identidade.
            \item Verificar se $\sigma \circ \tau$ é igual a $\tau \circ \sigma$ (comutatividade da composição).
            \item Encontrar a inversa explícita de $\sigma \circ \tau$ (se solicitado).
        \end{itemize}

    \subsection{Prova 3}
