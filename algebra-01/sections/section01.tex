\section{Princípios Básicos}

    \begin{itemize}[left=0.5cm, align=left, nosep]
        \item Assumimos os conceitos de \underline{\textbf{conjuntos}} e \underline{\textbf{subconjuntos}}. \\
        $\hookrightarrow$ $\in$ : "é elemento de" \\
        $\hookrightarrow$ $\mathbb{N}$ : Naturais \\
        $\hookrightarrow$ $\mathbb{Z}$ : Inteiros \\
        $\hookrightarrow$ $\mathbb{Q}$ : Racionais \\
        $\hookrightarrow$ $\mathbb{R}$ : Reais \\
        $\hookrightarrow$ $\mathbb{C}$ : Complexos 
    \end{itemize} 

    \subsection{Propriedades}

        \begin{itemize}
            \item Assumindo que esses conjuntos númericos tenham essas propriedades :  
        \end{itemize}

        \begin{enumerate}[label=\textbf{A\arabic*.}, align=left, leftmargin=2cm, nosep]
            \item $(x + y) + z = x + (y + z)$
            \item $0 + x = x$
            \item A equação $a + x = b$ possui uma única solução em $\mathbb{Z}$, se $a,b \in \mathbb{Z}$ (respectivamente, $\mathbb{Q},\mathbb{R}$ e $\mathbb{C}$).
            \item $x + y = y + x$
        \end{enumerate}

        \begin{enumerate}[label=\textbf{M\arabic*.}, align=left, leftmargin=2cm, nosep]
            \item $(xy)z = x (yz)$
            \item $1 \cdot x = x$
            \item A equação $ax = b$ possui uma única solução em $\mathbb{Q}$, se $a,b \in \mathbb{Q}$ (respectivamente, $\mathbb{R}$ e $\mathbb{C}$).
            \item $xy = yx$
        \end{enumerate}

        \hspace{1cm}\textbf{MA.} $x \cdot (y + z) = xy + xz$ e $(x + y) \cdot z = xy + xz$ (\textbf{Lei Distributiva}).

        \hspace{2cm}\fbox{Para $a \neq 0$. As soluções para \textbf{A3} e \textbf{M3} são respectivamente: $b-a$ e $\frac{b}{a}$.}

    \subsection{Definições}

        \begin{itemize}[left=0.5cm, align=left, nosep]
            \item \textbf{Anel:} Possui as propriedades \textbf{A1}, \textbf{A2}, \textbf{A3}, \textbf{A4} e \textbf{MA}.
            \item \textbf{Corpo:} Possui \underline{todas} as \textbf{noves} propriedades.
        \end{itemize}

    \subsection{Números Reais}

        \begin{itemize}[left=0.5cm, align=left, nosep]
            \item Aqui estão algumas inferências sobre os números reais ($\mathbb{R}$).
        \end{itemize} 

        \begin{enumerate}[left=1cm, align=left, nosep]
            \item É \underline{positivo} ($0 \geq $) ou \underline{negativo} ($0 \leq$).
            \item $0$ é \underline{positivo} e \underline{negativo}.
            \item $b \geq a \text{ }(\text{ou } a \leq b) \rightarrow a-b \geq 0$.
            \item $b > a \text{ }(\text{ou }  a < b) \rightarrow b \geq a \text{ e } b \ne a$.
        \end{enumerate}

    \subsection{Múltiplos e Divisores}

        \begin{itemize}[left=0.5cm, align=left, nosep]
            \item Se $a,b$ e $c$ são inteiros e \boxed{b = a \cdot c} \\ 
            $\hookrightarrow$ $b$ : É \underline{múltiplo} de $a$ \\  
            $\hookrightarrow$ $a$ \underline{divide} $b$ ou $a$ é \underline{divisor} de $b$. \\
            $\hookrightarrow$ \textbf{Notação} : $a|b$.
        \end{itemize}

        \begin{definicao}[: Par]
            $b$ é um inteiro \underline{\textbf{par}} se \underline{$2|b$}, caso contrário, $b$ é \underline{\textbf{ímpar}}
        \end{definicao}

        \begin{exemplo}[: Soma dos Divisores]
        $6 \Rightarrow 1 + 2 + 3 + 6 = 12$ \\
        $15 \Rightarrow 24$ \\
        $945 \Rightarrow 1920$
        \end{exemplo}

        \begin{questao}[(Números Perfeitos)]
        Existe um número ímpar $n$ tal que a soma dos seus divisores positivos seja igual a $2n$ ?
        \end{questao}

    \subsection{Princípio da Boa Ordenação e Indução}

        \begin{definicao}[: Princípio da Boa Ordenação (PBO)]
            Cada conjunto não vazio de interios positivos contém um \underline{menor elemento}.
        \end{definicao}

        \begin{exemplo}[: Prove que não existe um número inteiro $x$ tal que $0 < x < 1$.]
        De fato, se o conjunto de todos os inteiros x tais que $0 < x < 1$ fosse não vazio,
        o Princípio da Boa Ordenação é garantido que existiria um menor inteiro m tal que,
        $0 < m < 1$.

        Então, teriamos que $0 < m^2 < m < 1$; uma contradição
        \end{exemplo}

        \vspace{0.05cm}
        \noindent Uma formulação equivalente do \textbf{Princípio da Boa Ordenação} é o \textbf{Princípio da Indução Matemática}.

        \begin{definicao}[: Princípio da Indução Matemática (PIM)]
            Se uma senteça sobre um \underline{inteiro positivo} $n$ é verdadeira para $n = 0$, e se sua veracidade 
            para cada $n$ com \underline{$0 \leq n \leq N$} implica sua veracidade para \underline{$n = N$},então ela é verdadeira \underline{para todo $n \geq 0$}. 
        \end{definicao}

    \subsection{Exemplos}

    \begin{exemplo}[: Soma dos primeiros números naturais]
    Mostre que a igualdade $1+2+\dots+n=\tfrac{n\cdot(n+1)}{2}$ vale para cada $n\ge1$.
    \end{exemplo}

    De fato, para $n=1$ temos :
    \begin{align*}
    1 = \frac{1(1+1)}{2}
    \end{align*}

    Suponhamos que a igualdade seja válida para cada $n$ com $1 \leq n < N$.

    Daí,
    \begin{align*}
        1+2+\dotsc +(N-1)+N &= (1+2+\dotsc +N-1) + N \\
        &= \frac{(N-1) \cdot ((N-1) + 1)}{2} + N \\
        &= \frac{N^2 - N + 2N}{2} \\
        &= \frac{N \cdot (N+1)}{2} \\
    \end{align*}

    Logo, pelo Princípio da Indução Matemática a igualdade é válida para todo $n \ge 1$.

    \begin{exemplo}[]
    Demostre a validade de $2^n > 18(n+1)$ para cada $n \ge 8$.
    \end{exemplo}

    De fato, para $n = 8$ temos : 
    \begin{align*}
    2^8 = 256 > 162 = 18\cdot(9)
    \end{align*}

    Suponhamos que essa afirmação é válida para cada $n$ com $8 \leq n < N$.

    Daí,
    \begin{align*}
        2^{N} = 2^{N-1} \cdot 2 &> 18 \cdot (N - 1 + 1) \cdot 2 \\
        &> (18N) \cdot 2 = 18N + 18N \\
        &> 18N + 18 = 18 \cdot (N+1) \\
    \end{align*}

    Logo, pelo Princípio da Indução Matemática a afirmação é válida para cada $n \ge 8$.

    \vspace{0.05cm}

    \begin{exemplo}[: Sequências]
        \begin{equation*}
            \begin{alignedat}{2}
                a_n &=\; \begin{cases}
                    1, & \text{se } n = 1,\\
                    a_{n-1}+3, & \text{se } n \ge 2,
                \end{cases}
                &\qquad
                b_n &= a_1 + \dotsc + a_n \;=\; \sum_{k=1}^{n} a_k.
            \end{alignedat}
    \end{equation*}

        Para cada $n \ge 1$ mostre que :
        \begin{enumerate}[label=\roman*.]
            \item $a_{n} = 1 + 3 \cdot (n-1)$
            \item $b_{n} = \frac{3n^{2} - n}{2}$
        \end{enumerate}
        
    \end{exemplo}

    De fato, para $n = 1$ temos :

    \begin{align*}
    a_n = 1 + 3 \cdot (1 - 1) = 1
    \end{align*}

    Suponhamos que a afirmação seja válida para cada n com $1 \leq n < N$.

    Daí,
    \begin{align*}
    a_{N} &= a_{N-1} + 3 \\
    &= 1 + 3 \cdot ((N-1) - 1) + 3 \\
    &= 1 + 3 \cdot ((N-1) - 1 + 1) = 1 + 3 \cdot (N - 1) 
    \end{align*}

    Logo, pelo Princípio da Indução Matemática a afirmação é válida para todo $n \ge 1$.

    Por outro lado, para $n = 1$ temos : 
    \begin{align*}
    b_{N} = 1 = \frac{3(1)^2 - (1)}{2} \\
    \end{align*}

    Suponhamos que a afirmação seja válida para cada n com $1 \leq n < N$.

    Daí,
    \begin{align*}
    b_{N} &= a_1 + \dotsc + a_{N-1} + a_{N} \\
    &= \frac{3(N-1)^2 - (N-1)}{2} + 1 + 3 \cdot (N - 1) \\
    &= \frac{ 3N^{2} -6N + 3 -N + 1 + 2 + 6N + 3 }{2} \\
    &= \frac{3N^{2} - N}{2}
    \end{align*}

    Logo, pelo Princípio da Indução Matemática a afirmação é válida para cada $n \ge 1$.
