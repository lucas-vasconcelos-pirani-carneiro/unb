% --- Pacotes principais --- %
\usepackage[brazil]{babel}
\usepackage[utf8]{inputenc}
\usepackage[T1]{fontenc}
\usepackage{amsmath, amssymb}
\usepackage{enumitem}
% --- Fonte e layout otimizados para anotações ---
\usepackage[sfdefault]{sourcesanspro} % Fonte limpa e moderna
\usepackage[a4paper, left=2.3cm, right=2.3cm, top=2.8cm, bottom=2.8cm]{geometry}
\linespread{1.15} % ótimo espaçamento para leitura
\usepackage{xcolor}
\usepackage[most]{tcolorbox}
\usepackage{setspace}
\usepackage{tikz}
\usepackage{float}
\usepackage{colortbl}
\usepackage{array}
\usepackage{hyperref}
\usepackage{tocloft}
\usepackage[normalem]{ulem}
\usepackage{placeins}
\usepackage{fancyhdr}
\setlength{\headheight}{25pt}
\addtolength{\topmargin}{-10pt}

% --- Configuração de colunas ---
\newcolumntype{P}[1]{>{\centering\arraybackslash}p{#1}}

% --- Bibliotecas TikZ ---
\usetikzlibrary{trees, positioning}

% --- Estilo geral ---
\setstretch{1.15}
\setlength{\parskip}{0.5em}
\setlength{\parindent}{0pt}

% --- Espaçamento entre seções ---
\usepackage{titlesec}
\titlespacing*{\section}{0pt}{0.8em}{0.5em}
\titlespacing*{\subsection}{0pt}{0.6em}{0.3em}

% --- Configuração do sumário ---
\setlength{\cftbeforesecskip}{0.1em}
\setlength{\cftbeforesubsecskip}{0.05em}
\setlength{\cftbeforesubsubsecskip}{0.1em}

% --- Paleta preto e cinza ---
\definecolor{cinzaEscuro}{RGB}{40, 40, 40}
\definecolor{cinzaClaro}{RGB}{230, 230, 230}
\definecolor{cinzaMedio}{RGB}{120, 120, 120}
\definecolor{pretoFosco}{RGB}{15, 15, 15}

% --- Cores e estilo das seções ---
\usepackage{sectsty}
\sectionfont{\color{cinzaEscuro}\large\sffamily}
\subsectionfont{\color{cinzaMedio}\sffamily}
\subsubsectionfont{\color{cinzaMedio}\sffamily}

% --- Cabeçalho e rodapé ---
\pagestyle{fancy}
\fancyhf{}
\lhead{\textcolor{cinzaMedio}{\textbf{Algébra 01}}}
\rfoot{\textcolor{cinzaMedio}{\thepage}}
\renewcommand{\headrulewidth}{0.4pt}
\renewcommand{\footrulewidth}{0.4pt}
\renewcommand{\headrule}{\hbox to\headwidth{\color{cinzaClaro}\leaders\hrule height \headrulewidth\hfill}}
\renewcommand{\footrule}{\hbox to\headwidth{\color{cinzaClaro}\leaders\hrule height \footrulewidth\hfill}}

\fancypagestyle{plain}{%
  \fancyhf{}
  \lhead{\textcolor{cinzaMedio}{\textbf{Algébra 01}}}
  \rfoot{\textcolor{cinzaMedio}{\thepage}}
  \renewcommand{\headrulewidth}{0.4pt}
  \renewcommand{\footrulewidth}{0.4pt}
  \renewcommand{\headrule}{\hbox to\headwidth{\color{cinzaClaro}\leaders\hrule height \headrulewidth\hfill}}
  \renewcommand{\footrule}{\hbox to\headwidth{\color{cinzaClaro}\leaders\hrule height \footrulewidth\hfill}}
}

% --- Boxes estilizados com a paleta do documento ---
\usepackage[most]{tcolorbox}

% Box de Exemplo (cinza e preto fosco)
\newtcolorbox{exemplo}[1][]{
  colback=cinzaClaro!40!white,
  colframe=pretoFosco!80!black,
  fonttitle=\bfseries\sffamily,
  title=\textcolor{white}{Exemplo~#1},
  boxrule=0.6pt,
  arc=4pt,
  left=8pt, right=8pt, top=6pt, bottom=6pt,
  coltitle=pretoFosco
}

% Box de Questão (cinza escuro e médio)
\newtcolorbox{questao}[1][]{
  colback=cinzaClaro!30!white,
  colframe=cinzaEscuro!80!black,
  fonttitle=\bfseries\sffamily,
  title=\textcolor{white}{Questão~#1},
  boxrule=0.6pt,
  arc=4pt,
  left=8pt, right=8pt, top=6pt, bottom=6pt,
  coltitle=cinzaEscuro
}

% Box de Definição (cinza e preto fosco)
\newtcolorbox{definicao}[1][]{
  colback=cinzaClaro!40!white,
  colframe=pretoFosco!80!black,
  fonttitle=\bfseries\sffamily,
  title=\textcolor{white}{Definição~#1},
  boxrule=0.6pt,
  arc=4pt,
  left=8pt, right=8pt, top=6pt, bottom=6pt,
  coltitle=pretoFosco
}

% Box de Definição (cinza e preto fosco)
\newtcolorbox{teorema}[1][]{
  colback=cinzaClaro!40!white,
  colframe=pretoFosco!80!black,
  fonttitle=\bfseries\sffamily,
  title=\textcolor{white}{Teorema~#1},
  boxrule=0.6pt,
  arc=4pt,
  left=8pt, right=8pt, top=6pt, bottom=6pt,
  coltitle=pretoFosco
}

% --- Comando para número circundado ---
\usepackage{tikz}
\newcommand*\circled[1]{%
  \tikz[baseline=(char.base)]{
    \node[shape=circle,draw=cinzaMedio,inner sep=1pt, text=cinzaEscuro] (char) {#1};
  }%
}
