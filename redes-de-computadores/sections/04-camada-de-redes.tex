\section{Camada de Redes}

    %\begin{itemize}[left=0.5cm, align=left, nosep]
            %    \item \\
            %    \begin{itemize}[left=0.5cm, nosep, label=$\hookrightarrow$]
            %        \item \\
            %    \end{itemize}     
            %\end{itemize}
    \subsection{Introdução}

        \begin{itemize}[left=0.5cm, align=left, nosep]
            
            \item \textbf{Função principal:} fornece o transporte de segmentos dos hospedeiros de origem até os hospedeiros de destino.   
                \begin{itemize}[left=0.5cm, nosep, label=$\hookrightarrow$]
                    \item \textbf{Transmissor:} encapsula os segmentos em datagramas e encaminha para a camada de enlace; 
                    \item \textbf{Receptor:} entrega os segmentos ao protocolo da camada de transporte.
                \end{itemize}     
                
                \begin{center}
                    \includegraphics[width=0.5\textwidth]{img/cap-04/camada-de-redes-introd.png}
                \end{center}
            
            \item A camada de redes deve estar implementada em todos os dispositivos ao longo da rota entre origem e destino.
        
        \end{itemize}

        \subsubsection*{Roteadores}
            \begin{itemize}[left=0.5cm, align=left, nosep]
            
                \item \textbf{Função :}
                \begin{itemize}[left=0.5cm, nosep, label=$\hookrightarrow$]
                    \item Examinar os cabeçalhos dos datagramas IP para extrair o endereço de destino; 
                    \item Decidir para qual interface de saída o pacote será encaminhado.
                \end{itemize} 
                 
            \end{itemize}

        \subsubsection*{Funções Principais da Camada de Rede}
            \begin{itemize}[left=0.5cm, align=left, nosep]
                \item \textbf{Encaminhamento:} move pacotes de uma interface de entrada para a interface de saída apropriada;
                \item \textbf{Roteamento:} determina a rota da origem até o destino (definida por algoritmos de roteamento).
            \end{itemize}

            \textbf{Analogia:} Fazer uma viagem
            \begin{itemize}[left=0.5cm, align=left, nosep]
                \item \textbf{Encaminhamento:} decisão tomada em cada encruzilhada;
                \item \textbf{Roteamento:} processo de definir todo o caminho de origem até o destino.
            \end{itemize}

            \begin{center}
                \includegraphics[width=0.5\textwidth]{img/cap-04/viagem.png}
            \end{center}

        \subsubsection*{Plano de Dados}
            \begin{itemize}[left=0.5cm, align=left, nosep]
                \item Local por dispositivo, função implementada por roteador;
                \item Ao receber um datagrama, o roteador analisa o cabeçalho e decide o encaminhamento.
            \end{itemize}

            \begin{center}
                \includegraphics[width=0.4\textwidth]{img/cap-04/plano-de-dados.png}
            \end{center}

        \subsubsection*{Plano de Controle}
            \begin{itemize}[left=0.5cm, align=left, nosep]
                \item Lógica de toda a rede, determinando como datagramas são roteados de origem até destino;
                \item Implementações possíveis:
                
                \begin{enumerate}[left=0.5cm, align=left, nosep]
                    
                    \item \textbf{Algoritmos tradicionais de roteamento :} 
                        \begin{itemize}[left=0.5cm, nosep, label=$\hookrightarrow$]
                            \item Implementados nos roteadores; 
                            \item Cada roteador executa uma parte do algoritmo de forma distribuída.
                        \end{itemize} 
                    
                    \begin{center}
                        \includegraphics[width=0.5\textwidth]{img/cap-04/algorimo-roteamento.png}
                    \end{center}
                    
                    \item \textbf{SDN (Software Defined Networking) :} 
                        \begin{itemize}[left=0.5cm, nosep, label=$\hookrightarrow$]
                            \item Implementado em controladores remotos; 
                            \item Controladores calculam e instalam as tabelas de encaminhamento nos roteadores.
                        \end{itemize} 
                         
                    \begin{center}
                        \includegraphics[width=0.5\textwidth]{img/cap-04/sdn.png}
                    \end{center}
                
                \end{enumerate}

            \end{itemize}

        \subsubsection*{Modelo de Serviço de Redes para Datagramas}
            \begin{itemize}[left=0.5cm, align=left, nosep]
                \item \textbf{Serviços para datagramas individuais :}
                    \begin{itemize}[left=0.5cm, nosep, label=$\hookrightarrow$]
                        \item Garantia de entrega; 
                        \item Garantia de entrega com atraso menor que 40 ms; 
                        \item Garantia de integridade; 
                        \item Garantia de confiabilidade.
                    \end{itemize} 
                     
                \item \textbf{Serviços para fluxos de datagramas :}
                    \begin{itemize}[left=0.5cm, nosep, label=$\hookrightarrow$]
                        \item Entrega em ordem;
                        \item Garantia de taxa mínima de transferência; 
                        \item Restrição de variação de atraso (jitter).
                    \end{itemize} 
                     
            \end{itemize}

        \textbf{Modelo da Internet:} \textit{"Best Effort"} — não há garantias de entrega, temporização, ordenação ou taxa mínima de transmissão.

        \begin{table}[h!]
            \centering
            \renewcommand{\arraystretch}{1.3}
            \setlength{\tabcolsep}{3pt}
            \arrayrulecolor{blue}
            \begin{tabular}{|c|c|cccc|}
                \hline
                \textbf{Network Architecture} & \textbf{Service Model} &
                \multicolumn{4}{c|}{\textbf{Quality of Service (QoS) Guarantees?}} \\ \cline{3-6}
                & & \textbf{Bandwidth} & \textbf{Loss} & \textbf{Order} & \textbf{Timing} \\ \hline
                Internet & best effort & none & no & no & no \\ \hline
                ATM & Constant Bit Rate & Constant rate & yes & yes & yes \\ \hline
                ATM & Available Bit Rate & Guaranteed min & no & yes & no \\ \hline
                Internet & Intserv (RFC 1633) & yes & yes & yes & yes \\ \hline
                Internet & Diffserv (RFC 2475) & possible & possibly & possibly & no \\ \hline
            \end{tabular}
            \caption{Comparação de garantias de QoS entre diferentes arquiteturas e modelos de serviço.}
        \end{table}

        \subsubsection*{Reflexões sobre o Serviço “Best Effort”}
            \begin{itemize}[left=0.5cm, align=left, nosep]
                \item A simplicidade do mecanismo permitiu o crescimento da Internet;
                \item Na prática, há \textbf{superprovisionamento} de recursos para evitar congestionamentos;
                \item Há múltiplos mecanismos de \textbf{replicação e distribuição de conteúdo}, mantendo dados próximos dos usuários;
                \item Muitos serviços replicam conteúdo localmente, inclusive no provedor de acesso;
                \item O controle de congestionamento é eficiente em \textbf{serviços elásticos} — aplicações que usam toda a banda disponível.
            \end{itemize}

    \subsection{O tem dentro de um roteador ?}
        
        \subsubsection{Arquitetura geral de um roteador :}
        
            \begin{center}
                \includegraphics[width=0.45\textwidth]{img/cap-04/arquitetura-geral-roteador.png}
            \end{center}

        \subsubsection{Porta de Entrada :}
            
            \begin{center}
                \includegraphics[width=0.55\textwidth]{img/cap-04/porta-de-entrada.png}
            \end{center}

        \begin{itemize}[left=0.5cm, align=left, nosep]    
            \item \textbf{Caixa Verde (Camada Física):} recepção dos bits;
            \item \textbf{Azul (Camada de Enlace):} Ethernet;
            \item \textbf{Vermelho:} operações de checagem do endereço de destino do pacote (lookup); 
            
            \begin{itemize}[left=0.5cm, nosep, label=$\hookrightarrow$]
                \item Objetivo da interface: encaminhar o pacote para a matriz de comutação na velocidade de linha; 
                \item Enfileiramento dos pacotes para evitar perdas em caso de chegada de datagramas for superior à capacidade
                \item Decisão de encaminhamento :
                
                \begin{itemize}[left=0.5cm, nosep, label=$-$]
                    \item Tradicional: Encaminhamento é baseado somente no endereço de IP do destino contido no cabeçalho;
                    \item Generalizado: Encaminhamento é baseado em qualquer informação contida no cabeçalho do datagrama
                \end{itemize}
        
            \end{itemize}
        
        \end{itemize}

        \subsubsection*{Decisão de Encaminhamento :} 
    
            \begin{itemize}[left=0.5cm, align=left, nosep]
                \item O roteador possui uma tabela de encaminhamento que referencia endereços de destino e indica a interface de saída correspondente.
            
                \begin{center}
                    \includegraphics[width=0.45\textwidth]{img/cap-04/tabela-de-encaminhamento.png}
                \end{center}

                \item \textbf{Casamento de maior prefixo:} busca o endereço que melhor coincide com o endereço de destino do pacote na tabela.
            
                \begin{table}[h!]
                    \centering
                    \renewcommand{\arraystretch}{1.2}
                    \setlength{\tabcolsep}{6pt}
                    \arrayrulecolor{blue}

                    \begin{tabular}{|l|c|}
                        \hline
                        \textbf{Destination Address Range} & \textbf{Link Interface} \\ \hline
                        11001000 00010111 00010*** ******** & 0 \\ \hline
                        11001000 00010111 00011000 ******** & 1 \\ \hline
                        11001000 00010111 00011*** ******** & 2 \\ \hline
                        \textit{otherwise} & 3 \\ \hline
                    \end{tabular}

                    \caption{Tabela de encaminhamento com faixas de endereços de destino e interfaces de saída.}
                \end{table} 

                \textbf{\textcolor{blue}{Examples:}} \\[2pt]
                \texttt{11001000 00010111 00010110 10100001} \quad \textcolor{blue}{which interface? 1 } \\
                \texttt{11001000 00010111 00011000 10101010} \quad \textcolor{blue}{which interface? 2 }

                \item Geralmente, usa-se alta performance com TCAMs (memória ternária de conteúdo), permitindo o casamento em apenas um ciclo de clock.

            \end{itemize}

        \subsubsection*{Matrizes de Comutação} 

            \begin{itemize}[left=0.5cm, align=left, nosep]
                
                \item Transportam pacotes da interface de entrada até a interface de saída apropriada
                \item \textbf{Taxa de comutação :} Velocidade com que pacotes podem ser transferidos das entradas para as saídas
                    \begin{itemize}[left=0.5cm, nosep, label=$\hookrightarrow$]
                        \item Frequentemente medida como múltiplo da taxa de linha de entrada/saída; 
                        \item  Taxa ideal = $N \times R$ (onde $N$ = número de entradas e $R$ = taxa dos enlaces das interfaces). 
                    \end{itemize}
                    
                \begin{center}
                    \includegraphics[width=0.65\textwidth]{img/cap-04/taxa-de-comutacao.png}
                \end{center}

                \item \textbf{Formas de implementação :}
                \begin{center}
                    \includegraphics[width=0.65\textwidth]{img/cap-04/implementacao-matriz-de-comutacao.png}
                \end{center}

                \item \textbf{Memória :}  
                    \begin{itemize}[left=0.5cm, nosep, label=$\hookrightarrow$]
                        \item Roteadores da primeira geração 
                        \item Usam computador tradicional com duas placas de rede 
                        \item Pacote é copiado para a memória do sistema antes do encaminhamento 
                        
                        \item Problemas :
                        \begin{itemize}[left=0.5cm, nosep, label=$-$] 
                            \item Contenção de barramento (depois que utilizado um barramento ninguem pode mais utiliza-lo)
                            \item Passagem dupla do pacote pelo barramento, limitando a velocidade.
                        \end{itemize}
                    
                    \end{itemize}
                        
                \begin{center}
                    \includegraphics[width=0.45\textwidth]{img/cap-04/matriz-de-comutacao-memoria.png}
                \end{center}

                \item \textbf{Barramento :}
                    \begin{itemize}[left=0.5cm, nosep, label=$\hookrightarrow$]
                        \item Datagrama sai da porta de entrada para a porta de saída por um barramento compartilhado.
                        \item Contenção do Barramento : Limitação de velocidade pela banda do barramento.
                        \item Restrição de contenção do barramento, uma vez que tem um pacote trafegando no barramento ninguem pode mais trafegar 
                    \end{itemize}
                     
                    \begin{center}
                        \includegraphics[width=0.45\textwidth]{img/cap-04/matriz-de-comutacao-barramento.png}
                    \end{center}

                \item \textbf{Interconexão :} 
                    \begin{itemize}[left=0.5cm, nosep, label=$\hookrightarrow$]
                        \item \textbf{Barra cruzada :} Série de barramentos interconectados, permitindo algum nível de paralelismo; 
                        \item Continua ocorrendo contenção
                    \end{itemize}
                     
                    \begin{center}
                        \includegraphics[width=0.55\textwidth]{img/cap-04/barra-cruzada.png}
                    \end{center}

                \item Comutação em múltiplos estágios: consiste em switches \(n \times n\) compostos por vários estágios, estruturados como uma interconexão de matrizes de comutação menores. 
                    \begin{itemize}[left=0.5cm, nosep, label=$\hookrightarrow$]
                        \item Permite níveis de paralelismo. 
                        \item Fragmenta o datagrama em células de tamanho fixo, possibilitando a paralelização do envio dessas células.
                        \item Comuta as células através da matriz de comutação e remonta os datagramas na saída.
                    \end{itemize}
                     
                    \begin{center}
                        \includegraphics[width=0.45\textwidth]{img/cap-04/comutacao-em-multiplos-estagios.png}
                    \end{center}

                \item É possível escalar esse processo ainda mais parelizando as matrizes de comuração
                
                    \begin{center}
                        \includegraphics[width=0.65\textwidth]{img/cap-04/multiplos-estagios-escalavel.png}
                    \end{center}
            
            \end{itemize}

        \subsubsection*{Enfileiramento em Interfaces}    

            \begin{itemize}[left=0.5cm, align=left, nosep]
                \item Enfileramento na interface de entrada
                    \begin{itemize}[left=0.5cm, nosep, label=$\hookrightarrow$]
                        \item Se a matriz de comutação é mais lenta do que as portas de entrada combinadas $\rightarrow$ enfileramento ocorrerá na entrada de filas. 
                        \item \textbf{Head of Lines} (HOL) blocking : um datagrama enfileirado na frente da fila previne outros na fila de moverem para frente.    
                    \end{itemize}
                     
                    \begin{center}
                        \includegraphics[width=0.45\textwidth]{img/cap-04/hol.png}
                    \end{center}

                \item Enfileramento na interface de saida
                    \begin{itemize}[left=0.5cm, nosep, label=$\hookrightarrow$]
                        \item Buffering (Descarte): É necessario quando datagramas chegam da matriz de comutação mais rápidos do que a taxa de transmissao do enlace 
                        
                        \item Politica de descarte ? 
                        \begin{itemize}[left=0.5cm, nosep, label=$-$]
                            \item Datagramas podem ser perdidos devido ao congestionamento,falta de buffer
                        \end{itemize}
                        
                        \item Disciplinas de escalonamento : formas de lidar com a organização dos pacotes nessa fila de interface de saida para permitir priorizar trafego ou eventualmente reduzir o impacto do infileiramento na comunicação fim a fim     
                       
                        \begin{center}
                            \includegraphics[width=0.45\textwidth]{img/cap-04/enfileiramento-interface-saida.png}
                        \end{center}

                        \begin{center}
                            \includegraphics[width=0.45\textwidth]{img/cap-04/enfileiramento-interface-saida2.png}
                        \end{center}

                        \item Descarte quando a taxa de chegada via switch excede a velocidade de linha de saida 
                        \item Enfileiramento (atraso) e a perda devida o overflow(transbordo) do buffer da porta de saida

                        \item \textbf{Quanto de memoria é preciso se alocada para essa fila ?}
                        \begin{itemize}[left=0.5cm, nosep, label=$-$]
                            \item RFC 3429 : quantidade de buffer necessario para alocar será RTT vezes a capacidade do enlace C.
                            \item Recomendações recentes : com $n$ fluxos, $buffering = RTT * C / \sqrt{N}$
                            
                            \item Porém muito buffer pode causar atraso (praticularmente em roteadores domesticos)
                            \begin{itemize}[left=0.5cm, nosep, label=$\blacktriangleright$]
                                \item RTTs longos : desempenho ruim para aplicações de tempo real,
                                \item recordar que o atraso baseado no controle de congestionamento  
                            \end{itemize}
                        
                        \end{itemize}
                
                    \end{itemize}
                
                \item Manejamento do Buffer (AQM)
                    \begin{itemize}[left=0.5cm, nosep, label=$\hookrightarrow$]
                        \item Gerenciar o uso das filas
                        \item Descarte : qual pacote adicionar, e qual descarta quando o buffer está cheio
                            \begin{itemize}[left=0.5cm, nosep, label=$-$]
                                \item Descarta da cauda : descarta pacote chegando
                                \item Prioridade : descartar/remover baseado em uma prioridade 
                            \end{itemize}    
                    
                        \item Marcação : qual pacotes marcar para sinalixar um congestionamento (ECN,RED)

                    \end{itemize} 
                    
                    \begin{center}
                        \includegraphics[width=0.35\textwidth]{img/cap-04/manejamento-buffer.png}
                    \end{center}
               
            \end{itemize}
            
        \subsubsection*{Pacotes de Escalonamento :}
            \begin{itemize}[left=0.5cm, align=left, nosep]
                \item Decidir qual o proximo pacote que sera transmitido no enlace
                
                \item \textbf{FCFS} (Firt Come,FIrst Server): Pacotes são transmitidos em ordem atraves da porta de saida
                    \begin{itemize}[left=0.5cm, nosep, label=$\hookrightarrow$]    
                        \item FIFO
                    \end{itemize}

                \item \textbf{Prioridade} :
                    \begin{itemize}[left=0.5cm, nosep, label=$\hookrightarrow$]    
                        
                        \item Classifica os pacotes chegados enfileirado-os em classes de acordo com a prioridade.
                        \begin{itemize}[left=0.5cm, nosep, label=$-$] 
                            \item Qualquer campo do cabeçalho pode ser usado para a classificação
                        \end{itemize}  

                        \item Envia pacotes com maiores prioridade na fila que tem pacotes bufferizados  
                        \begin{itemize}[left=0.5cm, nosep, label=$-$] 
                            \item FCFS sem a classe de prioridade
                        \end{itemize}                     
                    
                    \end{itemize}
                     
                    \begin{center}
                        \includegraphics[width=0.45\textwidth]{img/cap-04/prioridade.png}
                    \end{center}
               
                \item \textbf{Round Robin} :
                    \begin{itemize}[left=0.5cm, nosep, label=$\hookrightarrow$] 
                        \item Classifica os pacotes chegados enfileirado-os em classes de acordo com a prioridade. 
                        \begin{itemize}[left=0.5cm, nosep, label=$-$]
                            \item Qualquer campo do cabeçalho pode ser usado para a classificação
                        \end{itemize}      
                        
                        \item Servidor ciclicamente,repetidamente escaneia as classes de filas, enviando um pacote completo de cada cada classe(se necessário) a cada turno. 
                    
                        \begin{center}
                            \includegraphics[width=0.45\textwidth]{img/cap-04/round-robin.png}
                        \end{center}
                    
                    \end{itemize}

                \item \textbf{Weighted Fair Queues} (WFQ) :
                    \begin{itemize}[left=0.5cm, nosep, label=$\hookrightarrow$]
                        \item Generalização para o round robin
                        \item Cada classe,$i$, tem um peso, $W_i$, e recebe uma quantidade ponderada de serviço em cada ciclo
                        \[
                            \frac{w_i}{\sum_{j}^{} w_j}  
                        \]  
                        \item Banda minima garantida para as classes de baixa prioridade 
                         
                    \end{itemize}    
                     
                    \begin{center}
                        \includegraphics[width=0.45\textwidth]{img/cap-04/wfq.png}
                    \end{center}
            
            \end{itemize} 

    \subsection{IP: Internet Protocol}
    
        \begin{itemize}[left=0.5cm, align=left, nosep]
            \item Protocolo \texttt{IP} : Define a padronização para troca de informações na camada de rede da Internet:
            \begin{itemize}[left=0.5cm, nosep, label=$\hookrightarrow$]
                \item Formato do datagrama
                \item Endereçamento
                \item Convenções para tratamento de pacotes
            \end{itemize}
            \item Protocolo \texttt{ICMP} : Trata da sinalização e troca de informações de controle na rede:
            \begin{itemize}[left=0.5cm, nosep, label=$\hookrightarrow$]
                \item Reporta erros
                \item Transporta informações de sinalização
            \end{itemize}
        \end{itemize}


        \begin{itemize}[left=0.5cm, align=left, nosep]
            \item Módulos da Camada de Rede
        \end{itemize}

        \begin{center}
            \includegraphics[width=0.45\textwidth]{img/cap-04/modulos-camada-de-rede.png}
        \end{center}

        \subsubsection*{Formato do Datagrama}

            \begin{center}
                \includegraphics[width=0.45\textwidth]{img/cap-04/formato-datagrama.png}
            \end{center}

        \subsubsection*{Endereçamento}
           
            \begin{itemize}[left=0.5cm, align=left, nosep]
                \item \textbf{Endereço IP}: Identificador de 32 bits associado a cada \underline{interface} de um sistema final (host) ou roteador.  
                \begin{itemize}[left=0.5cm, nosep, label=$\hookrightarrow$]
                    \item Uma máquina com \(n\) interfaces possui \(n\) endereços IP.
                    \item Um mesmo host pode ter múltiplos endereços IP.
                \end{itemize}

                \item \textbf{Interface}: Conexão entre sistemas finais/roteadores e a camada de enlace.  
                \begin{itemize}[left=0.5cm, nosep, label=$\hookrightarrow$]
                    \item Roteadores tipicamente possuem múltiplas interfaces.
                    \item Sistemas finais, uma ou duas interfaces.
                    \item Exemplos: rede sem fio (Wi-Fi), rede móvel (4G/5G), Bluetooth.
                \end{itemize}

                \begin{center}
                    \includegraphics[width=0.45\textwidth]{img/cap-04/introducao-endereco-ip.png}
                \end{center}

                \item \textbf{Formato do Endereço IP}:  
                \begin{itemize}[left=0.5cm, nosep, label=$\hookrightarrow$]
                    \item Organizado em quatro porções de 8 bits.
                    \item Representado na forma decimal com pontos separando cada octeto (ex: \texttt{192.168.0.1}).
                \end{itemize}

                \[
                    \textbf{223.1.1.1}
                    =
                    \underbrace{11011111}_{223} \;
                    \underbrace{00000001}_{1} \;
                    \underbrace{00000001}_{1} \;
                    \underbrace{00000001}_{1}
                \]
            
            \end{itemize}

            \textbf{Sub-redes}

                \begin{itemize}[left=0.5cm, align=left, nosep]
                    \item Conjunto de interconexões que vinculam sistemas finais e interfaces de roteadores.
                    \item Dispositivos que se conectam fisicamente sem precisar de um roteador intermediário pertencem à mesma sub-rede.
                    \item Exemplo: Rede composta por três sub-redes.

                    \begin{center}
                        \includegraphics[width=0.45\textwidth]{img/cap-04/introducao-endereco-ip.png}
                    \end{center}

                    \item \textbf{Estrutura do Endereço IP :}
                    \begin{itemize}[left=0.5cm, nosep, label=$\hookrightarrow$]
                        \item Parte da sub-rede: bits mais significativos, comuns aos dispositivos da mesma rede.
                        \item Parte do hospedeiro: bits menos significativos, distintos em cada interface.
                    \end{itemize}

                    \item \textbf{Definição de sub-redes :}
                    \begin{itemize}[left=0.5cm, nosep, label=$\hookrightarrow$]
                        \item Desvincula cada interface do host/roteador, criando “ilhas” ou redes isoladas.
                        \item Cada uma dessas redes é chamada de sub-rede.
                    \end{itemize}

                    \begin{center}
                        \includegraphics[width=0.45\textwidth]{img/cap-04/sub-redes.png}
                    \end{center}

                    \item \textbf{Padrões de endereçamento :}
                    \begin{itemize}[left=0.5cm, nosep, label=$\hookrightarrow$]
                        \item Bits do hospedeiro todos 0 $\rightarrow$ endereço da sub-rede.
                        \item Bits do hospedeiro todos 1 $\rightarrow$ endereço de broadcast (datagrama enviado a todas as interfaces da sub-rede).
                    \end{itemize}

                    \begin{center}
                        \includegraphics[width=0.45\textwidth]{img/cap-04/sub-redes2.png}
                    \end{center}
                
                \end{itemize}
                            
            \textbf{CIDR (Classless Inter-Domain Routing)}

            \begin{itemize}[left=0.5cm, align=left, nosep]
                \item Permite definir sub-redes com máscaras de tamanho arbitrário, sem as restrições das classes A, B e C. 
                \item Representação: \texttt{a.b.c.d/x}, onde \(x\) indica quantos dos bits mais significativos pertencem à parte da rede (prefixo).
                \item Exemplo: o endereço \texttt{192.168.10.0/24} indica que os 24 bits iniciais identificam a rede.
            \end{itemize}

            \begin{center}
                \includegraphics[width=0.45\textwidth]{img/cap-04/cidr.png}
            \end{center}

        \subsubsection*{Como conseguir um endereço IP?}

            \textbf{Existem, na verdade, duas questões:}
            \begin{enumerate}
                \item Como um sistema final consegue um endereço IP dentro de uma rede? --> O sistema faz parte da rede.
                \item Como uma rede obtém um endereço IP? --> A rede faz parte do endereço.
            \end{enumerate}

            \textbf{Como um sistema final consegue um endereço IP?}
            \begin{itemize}[left=0.5cm, align=left, nosep]
                \item Configuração manual : Administradores configuram manualmente o endereço nos arquivos da interface de rede.
                \item DHCP (Dynamic Host Configuration Protocol) : Obtém dinamicamente endereços de um servidor.
                    \begin{itemize}[left=0.5cm, nosep, label=$\hookrightarrow$]
                        \item Plug and play  
                    \end{itemize} 
            \end{itemize}

            \textbf{DHCP (Dynamic Host Configuration Protocol)}
            \begin{itemize}[left=0.5cm, align=left, nosep]
                \item Objetivo : Sistema final obtém dinamicamente um endereço IP ao entrar na rede.
                \item Protocolo de aplicação baseado em UDP.
                    \begin{itemize}[left=0.5cm, nosep, label=$\hookrightarrow$]
                        \item Pode renovar o endereço reservado.
                        \item Permite reutilização de endereços.
                        \item Oferece suporte a usuários móveis que entram e saem da rede.
                    \end{itemize} 
                \item Visão geral $\rightarrow$ Processo de comunicação entre cliente e servidor DHCP :
                    \begin{itemize}[left=0.5cm, nosep, label=$\hookrightarrow$]
                        \item Sistema final envia uma mensagem em broadcast para descobrir servidores DHCP disponíveis [DHCP Discover].
                        \item Servidor responde com mensagem \texttt{DHCP Offer}.
                        \item Sistema final solicita o endereço IP $\rightarrow$ mensagem \texttt{DHCP Request}.
                        \item Servidor confirma o endereço $\rightarrow$ mensagem \texttt{DHCP ACK}.
                    \end{itemize}
            \end{itemize}

            \begin{center}
                \includegraphics[width=0.45\textwidth]{img/cap-04/dhcp.png}
            \end{center}

            \texttt{Broadcast universal} : Todos os bits do endereço IP configurados com 1 (255.255.255.255).

            \begin{center}
                \includegraphics[width=0.45\textwidth]{img/cap-04/dhcp2.png}
            \end{center}

            \begin{itemize}[left=0.5cm, align=left, nosep]
                \item DHCP pode fornecer informações adicionais além do endereço IP :
                    \begin{itemize}[left=0.5cm, nosep, label=$\hookrightarrow$]
                        \item Endereço do primeiro salto (roteador padrão).
                        \item Nome e endereço IP do servidor DNS.
                        \item Máscara da sub-rede : Usada para determinar os limites da rede e a porção de endereçamento IP destinada aos dispositivos locais.
                    \end{itemize}    
                \end{itemize}

            \begin{center}
                \includegraphics[width=0.45\textwidth]{img/cap-04/dhcp-exemplo.png}
            \end{center}

            \begin{center}
                \includegraphics[width=0.45\textwidth]{img/cap-04/dhcp-exemplo2.png}
            \end{center}


            \textbf{Como uma rede obtém um endereço IP ?}
            \begin{itemize}[left=0.5cm, align=left, nosep]
                \item Cada sub-rede recebe um bloco de endereços a partir de um provedor de acesso à Internet (ISP).
                
                \[
                    \textbf{Bloco do ISP:} \quad
                    \uline{11001000 \; 00010111 \; 00010000} \; 00000000
                    \quad 200.23.16.0/20
                \]
                
                \item ISPs podem dividir esse bloco em sub-blocos menores.
            \end{itemize}

            \begin{center}
                \begin{tabular}{l l l l l l}
                    \textbf{Organization 0} & 11001000 & 00010111 & 00010000 & 00000000 & 200.23.16.0/23 \\
                    \textbf{Organization 1} & 11001000 & 00010111 & 00010010 & 00000000 & 200.23.18.0/23 \\
                    \textbf{Organization 2} & 11001000 & 00010111 & 00010100 & 00000000 & 200.23.20.0/23 \\
                    \textbf{...}            & .....    & .....    & .....    & ....     & \\
                    \textbf{Organization 7} & 11001000 & 00010111 & 00011110 & 00000000 & 200.23.30.0/23 \\
                \end{tabular}
            \end{center}

            \subsubsection*{Endereçamento Hierárquico}
            \begin{itemize}[left=0.5cm, align=left, nosep]
                \item O endereçamento hierárquico permite divulgação eficiente de informações de roteamento (agregação de rotas).
            \end{itemize}

            \begin{center}
                \includegraphics[width=0.45\textwidth]{img/cap-04/hierarquia-de-endereco.png}
            \end{center}

            \begin{itemize}[left=0.5cm, align=left, nosep]
                \item Quando uma organização troca de provedor, o endereço pode ser mantido.
                \item O roteamento usa o casamento de maior prefixo (rotas específicas).
            \end{itemize}

            \begin{center}
                \includegraphics[width=0.45\textwidth]{img/cap-04/endereco-hierarquico.png}
            \end{center}

        \subsubsection*{Endereçamento de IP : Últimos Comentários}
            \textbf{Como um ISP consegue um bloco de endereços ?}
            \begin{itemize}[left=0.5cm, align=left, nosep]
                \item ICANN (Internet Corporation for Assigned Names and Numbers) : Responsável por alocar blocos de endereços IP a grandes provedores.
                    \begin{itemize}[left=0.5cm, nosep, label=$\hookrightarrow$]
                        \item Coordena cinco registros regionais (RIRs).
                        \item Administra os servidores raiz do DNS.
                        \item Gerencia a delegação de domínios de topo (TLDs), como \texttt{.com}, \texttt{.edu}, etc.
                    \end{itemize}     
            \end{itemize}

            \textbf{Temos endereços IPv4(IP 32-bits) suficientes?}
            \begin{itemize}[left=0.5cm, align=left, nosep]
                \item A ICANN alocou o último bloco de endereços IPv4 em 2011.
                \item NAT prolonga a utilização do espaço IPv4.
                \item IPv6 possui espaço de endereçamento de 128 bits.
            \end{itemize}
    
        \subsubsection*{NAT: Network Address Translation}

            \begin{itemize}[left=0.5cm, align=left, nosep]
                \item Todos os dispositivos em uma rede local compartilham \underline{apenas um} endereço IPv4 do ponto de vista da Internet.

                \begin{center}
                    \includegraphics[width=0.45\textwidth]{img/cap-04/nat.png}
                \end{center}

                \item Dispositivos em uma rede local possuem um espaço de endereçamento IP privado de 32 bits, prefixos 10/8, 172.16/12 e 192.168/16 (válido apenas dentro da rede local).

                \item Vantagens : 
                \begin{itemize}[left=0.5cm, nosep, label=$\hookrightarrow$]
                    \item Apenas um endereço público é suficiente para todos os dispositivos internos.  
                    \item Mudança de endereços internos sem necessidade de informar a rede externa.  
                    \item Troca de provedor de Internet sem impacto no espaço interno.  
                    \item Segurança : Dispositivos internos não são diretamente acessíveis externamente.
                \end{itemize}     
                
                \item Implementação : Ocorre na borda da rede.
                \begin{itemize}[left=0.5cm, nosep, label=$\hookrightarrow$]
                    \item Datagramas de saída : substitui (endereço IP de origem, número da porta) por (endereço válido do NAT, novo número de porta)
                        \begin{itemize}[left=0.5cm, nosep, label=$-$]
                            \item Clientes e servidores externos respondem a esse novo par.
                        \end{itemize}
                    \item A tabela NAT mapeia (endereço IP de origem, número da porta) para (endereço NAT, nova porta). 
                    \item Datagramas de entrada : Realiza o processo inverso, substituindo (endereço NAT, nova porta) por (endereço interno, porta original).  
                \end{itemize}

                \begin{center}
                    \includegraphics[width=0.45\textwidth]{img/cap-04/nat2.png}
                \end{center}

                \item NAT é controverso
                \begin{itemize}[left=0.5cm, nosep, label=$\hookrightarrow$]
                    \item Roteadores deveriam operar apenas até a camada 3.
                    \item A escassez de endereços deveria ser resolvida pelo IPv6.
                    \item Viola o princípio fim-a-fim (endereços e portas são modificados pela camada de rede).
                    \item Atravessar NAT : Como um cliente externo se conecta a um servidor atrás do NAT ?
                \end{itemize}    

                \item Apesar disso, NAT continua amplamente usado :
                \begin{itemize}[left=0.5cm, nosep, label=$\hookrightarrow$]
                    \item Presente em redes domésticas, institucionais e em infraestruturas 4G/5G.
                \end{itemize}  
            
            \end{itemize}

        \subsubsection*{IPv6}
            \begin{itemize}[left=0.5cm, align=left, nosep]
                \item \textbf{Motivação inicial} : O espaço de endereçamento IPv4 (32 bits) foi completamente alocado.
                \item \textbf{Motivação adicional} :
                \begin{itemize}[left=0.5cm, nosep, label=$\hookrightarrow$]
                    \item Maior velocidade de processamento e encaminhamento, cabeçalho fixo de 40 bytes.  
                    \item Permite diferentes tratamentos de fluxo na camada de rede.
                \end{itemize}     
                
                \item Formato do datagrama IPv6
                \begin{center}
                    \includegraphics[width=0.4\textwidth]{img/cap-04/ipv6-formato}
                \end{center}

                \item Diferenças em relação ao IPv4 :
                \begin{itemize}[left=0.5cm, nosep, label=$\hookrightarrow$]
                    \item Sem campo de checksum (reduz o custo de processamento nos roteadores).  
                    \item Sem fragmentação ou remontagem de datagramas.  
                    \item Sem campo de opções, controle feito pelo cabeçalho de extensão (next header).
                \end{itemize} 

            \end{itemize}

        \subsubsection*{Transição do IPv64 para o IPv6}
            \begin{itemize}[left=0.5cm, align=left, nosep]
                \item Nem todos os roteadores podem ser atualizados simultaneamente
                
                \begin{itemize}[left=0.5cm, nosep, label=$\hookrightarrow$]
                    \item Sem “flag days” (substituição coletiva).  
                    \item Como a Internet lida com roteadores operando IPv4 e IPv6 simultaneamente?
                \end{itemize}     

                \item Tunelamento : Datagrama IPv6 encapsulado dentro de um datagrama IPv4 entre roteadores IPv4 (pacotes dentro de pacotes). 
                \begin{itemize}[left=0.5cm, nosep, label=$\hookrightarrow$]
                    \item Técnica usada também em outros contextos, como redes 4G/5G.
                \end{itemize} 

                \begin{center}
                    \includegraphics[width=0.4\textwidth]{img/cap-04/tunelamento.png}
                \end{center}

                \begin{center}
                    \includegraphics[width=0.4\textwidth]{img/cap-04/tunelamento-encapsulamento.png}
                \end{center}

                \begin{center}
                    \includegraphics[width=0.4\textwidth]{img/cap-04/tunelamento-encapsulamento2.png}
                \end{center}

                \begin{center}
                    \includegraphics[width=0.4\textwidth]{img/cap-04/tunelamento2.png}
                \end{center}

                \item Adoção do IPv6 : 
                \begin{itemize}[left=0.5cm, nosep, label=$\hookrightarrow$]
                    \item Google : 30\% dos clientes acessam via IPv6
                    \item NIST : Cerca de $\frac{1}{3}$ dos domínios do governo dos EUA têm suporte a IPv6.  
                    \item Processo de migração LONGO!! 
                    \begin{itemize}[left=0.5cm, nosep, label=$-$]
                        \item Mais de 25 anos 
                        \item Comparação: mudanças drásticas na camada de aplicação (WWW, redes sociais, streaming, jogos etc.).           
                    \end{itemize} 
                \end{itemize}    
                
                \begin{center}
                    \includegraphics[width=0.4\textwidth]{img/cap-04/adocao-ipv6.png}
                \end{center}

            \end{itemize}           
            
            


    \subsection{MiddleBox}

